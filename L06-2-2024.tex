

\title[概率论]{第七讲:条件概率、全概率公式与贝叶斯公式的应用}
%\author[张鑫 {\rm Email: xzhangseu@seu.edu.cn} ]{\large 张 鑫}
\institute[东南大学数学学院]{\large \textrm{Email: xzhangseu@seu.edu.cn} \\ \quad  \\
	\large 东南大学 \quad 数学学院 \\
	\vspace{0.3cm}
	%\trc{公共邮箱: \textrm{zy.prob@qq.com}\\
		% \hspace{-1.7cm}  密 \qquad 码: \textrm{seu!prob}}
}
\date{}


{\setbeamertemplate{footline}{}
	\begin{frame}
		\titlepage
	\end{frame}
}


%\subsection{条件概率}
%\begin{frame}{条件概率的引入}
%	\begin{itemize}
%		\item 有时除了 $P (\Omega)=1$ 的总前提之外,还会出现附加前提
%		\item 例如,抛掷一枚均匀的骰子一次,已知掷出的点数为奇数,要求求出点数大于 $1$ 的概率,那么此时 “已知掷出的点数为奇数” 就是一个附加前提
%		\item 有附加前提时的概率 $\Rightarrow$ 条件概率
%	\end{itemize}
%\end{frame}





\begin{frame}{条件概率的例子:男孩女孩概率问题}
	\begin{exam}
	(年长的是女孩 vs 至少一个女孩) 某家庭有两个孩子,已知至少有一个是女孩。两个孩子都是女孩的概率是多少?如果条件改为年长的孩子是女孩,那么两个都是女孩的概率又是多少?
	\end{exam}

	\begin{jieda}
	假设每个孩子都是女孩和男孩的可能性相同且不相关,那么
        \begin{align}
            &P (\mbox{都是女孩 | 至少有一个是女孩})\pause
            \\
			&=\frac{P (\mbox{都是女孩,至少有一个是女孩})}{P (\mbox{至少有一个是女孩})}\pause=\frac{1/4}{3/4}=1/3\pause\\
            &P (\mbox{都是女孩 | 年长的是女孩})\pause\\
            &=\frac{P (\mbox{都是女孩,年长的是女孩})}{P (\mbox{年长的是女孩})}\pause=\frac{1/4}{1/2}=1/2
        \end{align}
	\end{jieda}


\end{frame}

%

\begin{frame}{男孩女孩概率问题续:随机的一个孩子是女孩}
    \begin{exam}
        某家庭有两个孩子。随机遇到其中的一个,发现是女孩。给定这个信息后,两个孩子都是女孩的概率是多少?假设随机遇到两个孩子的可能性相同,且与性别无关.
    \end{exam}

    \begin{jieda}
        \begin{itemize}[<+-|alert@+>]
            \item 直观来看,结果应为 1/2.
            \item 令 $G_{1}$, $G_{2}$, $G_{3}$ 分别表示年长、 年幼、 随机的孩子是女孩这三个事件。由对称性可得: $P (G_{1})=P (G_{2})=P (G_{3})=1/2$
            \item 根据朴素概率的定义,或者独立性,可得: $P (G_{1}\cap G_{2})=1/4$
            \item 因此,$P (G_{1}\cap G_{2}|G_{3})=P (G_{1}\cap G_{2}\cap G_{3})/P (G_{3})=1/2$
            \item 又因为 $G_{1}\cap G_{2}\cap G_{3}=G_{1}\cap G_{2}$, 所以概率为 $1/2$.
        \end{itemize}
    \end{jieda}
	\pause
	\begin{rmk}
	假设一个强制性法律规定:如果一个男孩有姐妹则禁止他走出家门。那么这时 ``随机遇到的孩子是女孩" 就等价于 `` 至少有一个孩子是女孩”
	\end{rmk}

\end{frame}

%
\begin{frame}{男孩女孩概率问题续:冬天出生的女孩}
\begin{exam}
	某家庭有两个孩子,给定条件至少一个是女孩且在冬天出生,求两个孩子都是女孩的概率。假设四个季节出生的可能性相同且性别和季节是相互独立的.
\end{exam}
\pause

    \begin{jieda}
        由条件概率的定义,可得:
        \begin{align*}
            \ P (\mbox{两个都是女孩 | 至少有一个是冬天出生的女孩})\\
            =\frac{P (\mbox{两个都是女孩,至少有一个是冬天出生的女孩})}{P (\mbox{至少有一个是冬天出生的女孩})}
        \end{align*}\pause
		由于指定的孩子是在冬天出生的女孩的概率为 $1/8$, 所以,$$P (\mbox{至少有一个是在冬天出生的女孩}) = 1 - (7/8)^2.$$
    \end{jieda}
\end{frame}


		\begin{frame}{男孩女孩概率问题续:冬天出生的女孩}
			利用性别和季节是相互独立的假设,得到:
        \begin{align*}
            &P (\mbox{两个都是女孩,至少有一个是冬天出生的女孩})\\
            &=P (\mbox{两个都是女孩,至少有一个是冬天出生的})\\
            &=(1/4) P (\mbox{至少有一个是冬天出生的女孩})\\
            &=(1/4) (1-P (\mbox{所有孩子都不是在冬天出生的})
        \end{align*}
        合在一起得到,$$P (\mbox{两个都是女孩 | 至少有一个是冬天出生的女孩})=7/15$$
\end{frame}


\begin{frame}{结绳问题}
	\vspace{-0.15cm}
	\begin{exam}\
		$n$ 根绳 $2n$ 个头两两相接,求事件 $A=\{\mbox{恰好结成} n\mbox{个圈}\}$ 的概率.
	\end{exam}
	\pause
	\vspace{-0.4cm}
	\begin{itemize}[<+-|alert@+>]
		\item 设想 $2n$ 个头排成一行,规定将第 $2k-1$ 个头与第 $2k$ 个端头相接;
		\item 令 $B_i$ 表示第 $i$ 根绳的头与尾恰好相接,则 $A=B_1B_2\cdots B_n$;
		\item 若以 $n (A)$ 表示事件 $A$ 所包含的样本点个数,则易知 \pause
		\begin{eqnarray*}
			n(\Omega)=(2n)!, \pause n(B_1)=2n(2n-2)!\pause \Rightarrow P(B_1)=\dfrac{n(B_1)}{n(\Omega)}=\dfrac{1}{2n-1};\pause
		\end{eqnarray*}
		\item $P (B_2|B_1)$ 可看作 $n-1$ 根绳某根绳头尾相接的概率,类比 $n$ 根绳情形可得 $P (B_2|B_1)=\dfrac{1}{2 (n-1)-1}=\dfrac{1}{2n-3};$
		\item 同理可得 $P (B_k|B_1B_2\cdots B_{k-1})=\dfrac{1}{2[n-(k-1)]-1}=\dfrac{1}{2n-2k+1}, 3\leq k\leq n;$
		\item 利用乘法公式可得
		\begin{align*}
			P(A)&=\pause P(B_1B_2\cdots B_n)=\pause P(B_1)P(B_2|B_1)\cdots P(B_n|B_1\cdots B_{n-1})\pause\\
			&=\pause\dfrac{1}{(2n-1)!!}
		\end{align*}
	\end{itemize}


\end{frame}

\begin{frame}{取球问题}
	\begin{exam}
		在计算机中输入程序,让它自动完成如下操作:
		\begin{itemize}[<+-|alert@+>]
			\item 在 $1-\dfrac{1}{2^n}$ 时刻,往盒中放入标号 $10 (n-1)+1\sim 10n$ 的 $10$ 个球,同时取出标号为 $10 (n-1)+1$ 的球,$n\geq 1$;
			\item 在 $1-\dfrac{1}{2^n}$ 时刻,往盒中放入标号 $10 (n-1)+1\sim 10n$ 的 $10$ 个球,同时取出标号为 $n$ 的球,$n\geq 1$;
			\item 在 $1-\dfrac{1}{2^n}$ 时刻,往盒中放入标号 $10 (n-1)+1\sim 10n$ 的 $10$ 个球,同时随机地从盒中取出一个球,$n\geq 1$.
		\end{itemize}
	\pause
		则在时刻 $1$, 盒中的球数结果如下
		\begin{itemize}[<+-|alert@+>]
			\item 盒子中有无穷多个球;
			\item 盒子变为空的;
			\item 盒子变为空的概率等于 $1$?
		\end{itemize}
	\end{exam}
\end{frame}

\begin{frame}{取球问题}
	\jieda\
	\begin{itemize}[<+-|alert@+>]
		\item 记 $E$=\{在时刻 $1$ 时盒子变空 \};
		\item $\overline{E}$=\{在时刻 $1$ 时盒中有球未被取出 \};
		\item 记 $A_k$=\{在时刻 $1$ 时 $k$ 号球仍在盒中未被取出 \}, 则 \pause$\overline{E}=\bigcup\limits_{k=1}^{\infty} A_k$;\pause
		\item 由概率的次可加性知
		$$ P( \overline{E})= P\big(\bigcup_{k=1}^{\infty}A_k\big)\leq\sum_{k=1}^{\infty} P(A_k);$$
		\item 为证 $ P (\overline{E})=0$,只需证明 $P (A_k)=0,\,k\geq 1$;
		\item 由于证法类似,仅以证明 $ P (A_1)=0$ 为例;
	\end{itemize} %,则 $ \overline{E}$=\{在时刻 $1$ 时盒中有球未被取出 \}。为证 $ P (E)=1$,只需证明 $ P ( \overline{E})=0$。


\end{frame}

\begin{frame}{取球问题}
	\begin{itemize}[<+-|alert@+>]
	\item 记 $B_n$=\{在 $1-\dfrac{1}{2^n}$ 时刻 $1$ 号球未被取出 \},\pause 易知 $A_1=\bigcap\limits_{n=1}^{\infty} B_n$;\pause
	\item 令 $C_m=\bigcap\limits_{n=1}^{m} B_n$, 则有 \pause
	\[C_m=\bigcap\limits_{n=1}^{m} B_n\supset\bigcap\limits_{n=1}^{m+1} B_n=C_{m+1},\mbox{ 且 } \lim_{m\rightarrow\infty} C_m=A_1;\]\pause
	\item 由概率的上连续性知 $$ P (A_1)= P\big (\lim_{m\rightarrow\infty} C_m\big)=\lim_{m\rightarrow\infty} P\big (C_m\big);$$
	\item 下面求 $P (C_m)$, 即 $P\big (\bigcap_{n=1}^{m} B_n\big)=\prod_{n=1}^{m}\frac{9n}{9n+1}=\prod_{n=1}^{m}\left (1-\frac{1}{9n+1}\right)$;
	\begin{itemize}[<+-|alert@+>]
		\item $B_1$: 在 $1/2$ 时刻,盒中 $10$ 个球,$1$ 号球未被取出,故 \pause $ P (B_1)=\frac{9}{10}$;\pause
		\item $B_2$: 在 $3/4$ 时刻,盒中 $19$ 个球,$1$ 号球未被取出,故 \pause $ P (B_2|B_1)=\frac{18}{19}$;\pause
		\item $B_n$: $ P(B_n|B_1B_2\cdots B_{n-1})=\frac{9n}{9n+1}.$
	\end{itemize}
\item $ P (A_1)=\prod_{n=1}^{\infty}\big (1-\frac{1}{9n+1}\big)=0.$ 同理可证 $ P (A_k)=0,\,k=2,3,\cdots$.
\end{itemize} %,则 $ \overline{E}$=\{在时刻 $1$ 时盒中有球未被取出 \}。为证 $ P (E)=1$,只需证明 $ P ( \overline{E})=0$。



\end{frame}

%\begin{frame}{取球问题}
%	在时刻 $\dfrac{3}{4}$ 时,盒中共有 $19$ 个球,若 $1$ 号球仍未被取出,则在 $B_1$ 发生的条件下还有 $B_2$ 发生,此时有 $18$ 种取球方式,由于面对的是变化了的概率空间,故按无条件概率的求法计算出 $$ P (B_2|B_1)=\frac{18}{19}.$$
%	一般地,有 $$ P (B_n|B_1B_2\cdots B_{n-1})=\frac{9n}{9n+1}.$$
%	于是按照概率的乘法定理得到 $$ P\left (\bigcap_{n=1}^{m} B_n\right)=\prod_{n=1}^{m}\frac{9n}{9n+1}=\prod_{n=1}^{m}\left (1-\frac{1}{9n+1}\right).$$
%\end{frame}
%
%\begin{frame}{取球问题}
%	于是就有 $$ P (A_1)=\lim_{m\rightarrow\infty} P\left (\bigcap_{n=1}^{m} B_n\right)=\prod_{n=1}^{\infty}\left (1-\frac{1}{9n+1}\right).$$
%	由于 $$\sum_{n=1}^{\infty}\frac{1}{9n+1}=\infty,$$ 所以由无穷乘积发散的判别准则,知 $$ P (A_1)=\prod_{n=1}^{\infty}\left (1-\frac{1}{9n+1}\right)=0.$$
%	同理可证 $ P (A_k)=0,\,k=2,3,\cdots$。综合上述,就证出了在时刻 $1$ 时,盒子变为空的概率等于 $1$。
%\end{frame}
%
%\begin{frame}
%	\begin{itemize}
%		\item 上一题的解答需要清晰正确的转换思路:$E\rightarrow  \overline{E}\rightarrow A_1\rightarrow\bigcap\limits_{n=1}^{m} B_n$
%		\item 概率论中的许多问题都可以用罐中取球的模型来描述
%		\item 下一例出现的罐子模型是所谓有后效的模型,可用来粗略描述流行病的传播规律
%	\end{itemize}
%\end{frame}








\begin{frame}
	\frametitle{罐子模型(波利亚模型)}
	\begin{exam}
		设罐中有 $b$ 个黑球,$r$ 个红球,每次随机的取出一球,取出后将原球放回,还加进 $c$ 个同色球和 $d$ 个异色球。记 $B_i:=\{\mbox{第} i\mbox{次取出的是黑球}\}, R_j:=\{\mbox{第} j\mbox{次取出的是红球}\}$. 若连续从罐子中取出三个球,其中有两个红球,一个黑球,则由乘法公式可得
		\begin{eqnarray*}
			P(B_1R_2R_3)&=&P(B_1)P(R_2|B_1)P(R_3|B_1R_2)\\
			&=&\frac{b}{b+r}\cdot\frac{r+d}{b+r+c+d}\cdot\frac{r+d+c}{b+r+2c+2d}\\
			P(R_1B_2R_3)&=&P(R_1)P(B_2|R_1)P(R_3|R_1B_2)\\
			&=&\frac{r}{b+r}\cdot\frac{b+d}{b+r+c+d}\cdot\frac{r+d+c}{b+r+2c+2d}\\
			P(R_1R_2B_3)&=&P(R_1)P(R_2|R_1)P(B_3|R_1R_2)\\
			&=&\frac{r}{b+r}\cdot\frac{r+c}{b+r+c+d}\cdot\frac{b+2d}{b+r+2c+2d}
		\end{eqnarray*}
		显然以上概率与黑球在第几次抽取有关.
	\end{exam}
\end{frame}
\begin{frame}
	\frametitle{罐子模型(波利亚模型)}
	\vspace{-0.3cm}
	\begin{itemize}[<+-|alert@+>]
		\item 当 $c=-1, d=0$ 时,即为不返回抽样。此时前次抽取结果会影响后次抽取结果,但只要抽取的黑球与红球个数确定,则概率不依赖其抽出球的次序,都是一样的.
		{\small\begin{eqnarray*}
				P(B_1R_2R_3)= P(R_1B_2R_3)=P(R_1R_2B_3)=\frac{br(r-1)}{(b+r)(b+r-1)(b+r-2)}.
		\end{eqnarray*}}
		\item 当 $c=0,d=0$ 时,即为返回抽样。此时前次抽取结果不会影响后次抽取结果,故上述三个概率相等且都等于
		{\small\begin{eqnarray*}
				P(B_1R_2R_3)= P(R_1B_2R_3)=P(R_1R_2B_3)=\frac{br^2}{(b+r)^3}.
		\end{eqnarray*}}
		\item 当 $c>0,d=0$ 时,称为传染病模型。此时每次取出球后会增加下一次取出同色球的概率,或换言之,每发现一个传染病患者,以后都会增加再传染的概率。与前两种情况一样,三个概率都等于
		{\small\begin{eqnarray*}
				P(B_1R_2R_3)= P(R_1B_2R_3)=P(R_1R_2B_3)=\frac{br(r+c)}{(b+r)(b+r+c)(b+r+2c)}.                                                        \end{eqnarray*}}

	\end{itemize}
\end{frame}
\begin{frame}
	\frametitle{罐子模型(波利亚模型)}
	\begin{itemize}[<+-|alert@+>]
		\item 从上面的结果可以看出,只要 $d=0$,以上三个概率都相等,即只要抽取的黑球与红球的个数确定,则概率不依赖于抽出黑红球的次序.
		\item 当 $c=0,d>0$ 时,称为安全模型。此模型可解释为:每当事故发生了 (当红球被取出),安全工作就抓紧一些,下次再发生事故的概率就会减少,而当事故没有发生时 (黑球被取出),安全工作就放松一些,下次再发生事故的概率就会增大,此时,上述三个概率分别为
		{\small\begin{eqnarray*}
				P(B_1R_2R_3) &=&\frac{b}{b+r}\cdot\frac{r+d}{b+r+d}\cdot\frac{r+d}{b+r+2d}\\
				P(R_1B_2R_3)&=&\frac{r}{b+r}\cdot\frac{b+d}{b+r+d}\cdot\frac{r+d}{b+r+2d}\\
				P(R_1R_2B_3) &=&\frac{r}{b+r}\cdot\frac{r}{b+r+d}\cdot\frac{b+2d}{b+r+2d}
		\end{eqnarray*}}
	\end{itemize}

\end{frame}

\begin{frame}{罐子模型(波利亚模型)}
	\begin{exam}
	设罐中有 $b$ 个黑球,$r$ 个红球,每次随机取出一球后将原球放回并加进 $c$ 个同色球,如此反复进行。试证明:在前 $n=n_1+n_2$ 次取球中,取出了 $n_1$ 个红球和 $n_2$ 个黑球的概率为
		$$C_n^{n_1}\frac{a(a+c)(a+2c)\cdots(a+n_1c-c)b(b+c)(b+2c)\cdots(b+n_2c-c)}{(a+b)(a+b+c)(a+b+2c)\cdots(a+b+nc-c)}.$$
	\end{exam}
%	\begin{jieda}
%		记 $A_k$=\{第 $k$ 次取球时取出白球 \}, 于是 $A_k^c$=\{第 $k$ 次取球时取出黑球 \}. 采用逐个考虑被改变了的概率空间的方法,不难利用乘法定理求得
%		\begin{align*}
%			& P(A_1\cdots A_{n_1}A_{n_1+1}^c\cdots A_n^c)\\=&\frac{a(a+c)(a+2c)\cdots(a+n_1c-c)b(b+c)(b+2c)\cdots(b+n_2c-c)}{(a+b)(a+b+c)(a+b+2c)\cdots(a+b+nc-c)}.
%		\end{align*}
%	\end{jieda}
\end{frame}


%%% Local Variables:
%%% mode: latex
%%% TeX-master: t
%%% End:
