

\title[概率论]{第六讲:条件概率与事件的独立性}
%\author[张鑫 {\rm Email: xzhangseu@seu.edu.cn} ]{\large 张 鑫}
\institute[东南大学数学学院]{\large \textrm{Email: xzhangseu@seu.edu.cn} \\ \quad  \\
	\large 东南大学 \quad 数学学院 \\
	\vspace{0.3cm}
	%\trc{公共邮箱: \textrm{zy.prob@qq.com}\\
		% \hspace{-1.7cm}  密 \qquad 码: \textrm{seu!prob}}
}
\date{}


{\setbeamertemplate{footline}{}
	\begin{frame}
		\titlepage
	\end{frame}
}


%\subsection{条件概率}
%\begin{frame}{条件概率的引入}
%	\begin{itemize}
%		\item 有时除了 $P (\Omega)=1$ 的总前提之外,还会出现附加前提
%		\item 例如,抛掷一枚均匀的骰子一次,已知掷出的点数为奇数,要求求出点数大于 $1$ 的概率,那么此时 “已知掷出的点数为奇数” 就是一个附加前提
%		\item 有附加前提时的概率 $\Rightarrow$ 条件概率
%	\end{itemize}
%\end{frame}
\subsection{条件思考的重要性}%%%%%%%%%%%%%%%%%%%%%%%%%%%%%%%%%%%%%%%%%%%

\begin{frame}{条件思考的重要性}
	\begin{itemize}[<+-|alert@+>]
		\item 条件概率说明了如何以合乎逻辑且相一致的方式将证据纳入人们对世界的理解当中.
        \item 添加条件是一种非常有用的解决问题的策略.
        \item 条件既可以作为更新判断依据的方法,也是解决问题的策略.
        \item 条件是统计学的灵魂.
	\end{itemize}

\end{frame}

\subsection{条件概率的定义和直观解释}%%%%%%%%%%%%%%%%%%%%%%%%%%%%%%%

\begin{frame}{条件概率的定义}
	\begin{defi}
		设 $(\Omega,\mathcal{F},P)$ 为概率空间. $A\in\mathcal{F}$, $B\in\mathcal{F}$, 若 $P (B)>0$, 则称 $$P (A|B)=\frac{P (AB)}{P (B)}$$ 为 ``在已知 $B$ 发生的情况下,$A$ 发生的 \textcolor{red}{条件概率}", 简称条件概率.
	\end{defi}
	\pause
	\vspace{0.4cm}
	\begin{rmk} 条件概率的两种计算方法:
		\begin{itemize}[<+-|alert@+>]
			\item 原则性方法: $P (A|B)=\dfrac{P (AB)}{P (B)}$
			\item 把 $B$ 作为样本空间看待 (经常显得非常方便): $P (A|B)=\dfrac{|AB|}{|B|}$
		\end{itemize}
	\end{rmk}
\end{frame}
\begin{frame}
	\frametitle{条件概率的性质}
	\begin{thm}
		条件概率是概率,即若设 $P (B)>0$, 则
		\begin{itemize}[<+-|alert@+>]
			\item 非负性: $P (A|B)\geq 0, \forall A\in\mathcal{F}$;
			\item 规范性: $P (\Omega|B)=1$;
			\item 可列可加性:若 $A_n\in \mathcal{F}, n=1,2,\cdots,$ 互不相容,则
			\begin{eqnarray*}
				P(\cup_{n=1}^\infty A_n|B)=\sum_{n=1}^\infty P(A_n|B)
			\end{eqnarray*}

		\end{itemize}
	\end{thm}

	\pause \zheng 由条件概率的定义易证非负性与规范性。下面说明可列可加性。假设 $A_n,n=1,\cdots,$ 互不相容,则 $A_nB, n=1,\cdots,$ 也互不相容。从而
	\begin{eqnarray*}
		P(\cup_{n=1}^\infty A_n|B)&=&\frac{P((\cup_{n=1}^\infty A_n)B)}{P(B)}=\frac{P(\cup_{n=1}^\infty (A_nB))}{P(B)}\\
		&=&\sum_{n=1}^\infty \frac{P(A_nB)}{P(B)}=\sum_{n=1}^\infty P(A_n|B).
	\end{eqnarray*}
\end{frame}


\begin{frame}
	\frametitle{乘法公式}
	\begin{thm}[乘法公式]
		\begin{itemize}[<+-|alert@+>]
			\item 若 $P (A)>0$, 则 %\vspace{-0.85cm}
			\begin{eqnarray}
				\label{eq:multiformtwo}
				P(AB)=P(A)P(B|A)
			\end{eqnarray}
			\item 若 $P (A_1A_2\cdots A_{n-1})>0$, 则
			\begin{eqnarray}
				\label{eq:multiformmul}
				P(A_1A_2\cdots A_n)=P(A_1)P(A_2|A_1)P(A_3|A_1A_2)\cdots P(A_n|A_1A_2\cdots A_{n-1})
			\end{eqnarray}

		\end{itemize}
	\end{thm}
	\pause \zheng  由条件概率的定义,移项即得 \eqref{eq:multiformtwo} 式,下证 \eqref{eq:multiformmul}. \pause 因为
	\begin{eqnarray*}
		P(A_1)\ge P(A_2)\ge\cdots\ge P(A_1A_2\cdots A_{n-1})>0
	\end{eqnarray*}
	故 \eqref{eq:multiformmul} 式中条件概率均有意义。从而由两个事件的乘法公式可得
	\pause \begin{eqnarray*}
		&&P(A_1A_2\cdots A_n)=P(A_1A_2\cdots A_{n-1})P(A_n|A_1A_2\cdots A_{n-1})\pause \\
		&&=P(A_1\cdots A_{n-2})P(A_{n-1}|A_1\cdots A_{n-2})P(A_n|A_1A_2\cdots A_{n-1})\pause \\
		&&=\cdots =P(A_1)P(A_2|A_1)P(A_3|A_1A_2)\cdots P(A_n|A_1A_2\cdots A_{n-1})
	\end{eqnarray*}

\end{frame}



\begin{frame}{条件概率的例子}
\begin{exam}
	(两张牌) 洗好一副标准扑克后。从中随机抽取两张牌,无放回地一次抽一张。设 $A$ 事件表示第一张牌为红桃,事件 $B$ 表示第二张牌为红色。求 $P ( A | B )$ 和 $P ( B | A )$.
\end{exam}

\begin{jieda}
  由题意易知
  \begin{align*}
	P(A\cap B) &=\frac{13 \cdot 25}{52 \cdot 51}=\frac{25}{204}\pause \\
	P(B)&=\frac{26 \cdot 51}{52 \cdot 51}=\frac{1}{2} \\
	P(A) &=\frac{1}{4}
  \end{align*}
  \pause
  从而,
  \begin{align*}
	P(A|B) &=\frac{P(A\cap B)}{P(B)}=\frac{25/204}{1/2}=\frac{25}{102}\\
	P(B|A) &=\frac{P(B \cap A)}{P(B)}=\frac{25/204}{1/4}=\frac{25}{51}
  \end{align*}



\end{jieda}

\end{frame}
\begin{frame}{条件概率的注}
\begin{itemize}[<+-|alert@+>]
\item  注意哪些事件放在竖线的哪一边是非常重要的,具体来说就是 $P (A|B)\neq P (B|A)$.
\item 无论 $P (A|B)$ 还是 $P (B|A)$ 都是有意义的 (直观上或数学上):
\begin{itemize}[<+-|alert@+>]
   \item 牌抽取的时间顺序并不能决定出现何种条件概率.
   \item 在计算条件概率时,我们考虑的是一个事件给另一个事件带来的信息,而不是一个事件是否导致了另一个事件.
\end{itemize}

\item  此外,也可以通过条件概率的直接解释得出 $P (B|A)=25/51$:
\begin{itemize}[<+-|alert@+>]
	\item 如果第一张抽的牌为红桃,那么剩下的牌就由 25 张红色牌和 26 张黑色牌组成 (所有牌被下一次抽中的可能性是相同的)
	\item 所以抽取一张红牌的条件概率是 $25/(25+26)=25/51$.
\end{itemize}
\end{itemize}

\end{frame}

\subsection{条件概率的应用}
\begin{frame}{条件性思维谬误:控方证人的错误}
\begin{exam}{\tc (控方证人的错误)} 1988 年,Sally Clark 由于她的两个孩子在出生不久便死亡,因而被指控谋杀幼童。在审讯期间,控方的一个专家证人证实
\begin{itemize}[<+-|alert@+>]
\item 新生儿因婴儿猝死综合症 (SIDS) 而死亡的概率为 $1/8500$
\item 所以两个新生儿由于婴儿猝死综合症死亡的概率为 $(1/8500)^2$, 大约为 7300 万分之一
\item 因此,他认为 Clark 清白的概率仅为 7300 万分之一.
\end{itemize}
\end{exam}
\vspace{0.2cm}

\pause
\begin{jieda} 这个推理过程至少有两个问题
\begin{itemize}[<+-|alert@+>]
\item 一个家庭内部成员之间死于 SIDS 是否相互独立?
\begin{itemize}[<+-|alert@+>]
\item 家庭内部成员之间死于 SIDS 相互独立时,“第一个孩子死于 SIDS” 且 “第二个孩子也死于 SIDS” 的概率是相应的两个事件概率相乘
\item 如果遗传因素或其他家庭特有的风险因素导致某些家庭内的所有新生儿面临 SIDS 的风险增加,这种独立性就不再成立了.
\end{itemize}
\end{itemize}
\end{jieda}
\end{frame}

\begin{frame}{条件性思维谬误:控方证人的错误}
\begin{itemize}[<+-|alert@+>]
\item 这个所谓的专家将两个不同的条件概率混淆了: $P (\mbox{清白 | 证据})$ 和 $P (\mbox{证据 | 清白})$ 是不一样的.
\begin{itemize}[<+-|alert@+>]
\item  专家 1 声称:如果在被告人是清白的情况下,两个孩子死亡的概率很低;那就是说 $P (\mbox{证据 | 清白})$ 非常小.
\item 但人们感兴趣的是,给定现在所有的证据 (孩子均死) 条件下,报告人仍清白的概率,即 $P (\mbox{清白 | 证据})$.
\item 由贝叶斯准则可知,$$P (\mbox{清白 | 证据})=\frac{P (\mbox{证据 | 清白}) P (\mbox{清白})}{P (\mbox{证据})},$$

\item 所以为了计算 $P (\mbox{清白 | 证据})$, 这里需要考虑 $P (\mbox{清白})$, 也就是被告清白的先验概率。这个概率是很高的;
\item 虽然 SIDS 造成两个婴儿死亡是很罕见的,但是蓄意杀害两个婴儿的情况也很少见!
\item 基于现有证据的后验概率是对很低的 $P (\mbox{证据 | 清白})$ 和很高的 $P (\mbox{清白})$ 的一个平衡。专家的结果 $(1/8500)^2$ 是有问题的,它只是整个计算式中的一部分.
\end{itemize}
\end{itemize}
\end{frame}



\begin{frame}{条件概率的例子:男孩女孩概率问题}
	\begin{exam}
	(年长的是女孩 vs 至少一个女孩) 某家庭有两个孩子,已知至少有一个是女孩。两个孩子都是女孩的概率是多少?如果条件改为年长的孩子是女孩,那么两个都是女孩的概率又是多少?
	\end{exam}

	\begin{jieda}
	假设每个孩子都是女孩和男孩的可能性相同且不相关,那么
        \begin{align}
            &P (\mbox{都是女孩 | 至少有一个是女孩})\pause
            \\
			&=\frac{P (\mbox{都是女孩,至少有一个是女孩})}{P (\mbox{至少有一个是女孩})}\pause=\frac{1/4}{3/4}=1/3\pause\\
            &P (\mbox{都是女孩 | 年长的是女孩})\pause\\
            &=\frac{P (\mbox{都是女孩,年长的是女孩})}{P (\mbox{年长的是女孩})}\pause=\frac{1/4}{1/2}=1/2
        \end{align}
	\end{jieda}


\end{frame}

%

\begin{frame}{男孩女孩概率问题续:随机的一个孩子是女孩}
    \begin{exam}
        某家庭有两个孩子。随机遇到其中的一个,发现是女孩。给定这个信息后,两个孩子都是女孩的概率是多少?假设随机遇到两个孩子的可能性相同,且与性别无关.
    \end{exam}

    \begin{jieda}
        \begin{itemize}[<+-|alert@+>]
            \item 直观来看,结果应为 1/2.
            \item 令 $G_{1}$, $G_{2}$, $G_{3}$ 分别表示年长、 年幼、 随机的孩子是女孩这三个事件。由对称性可得: $P (G_{1})=P (G_{2})=P (G_{3})=1/2$
            \item 根据朴素概率的定义,或者独立性,可得: $P (G_{1}\cap G_{2})=1/4$
            \item 因此,$P (G_{1}\cap G_{2}|G_{3})=P (G_{1}\cap G_{2}\cap G_{3})/P (G_{3})=1/2$
            \item 又因为 $G_{1}\cap G_{2}\cap G_{3}=G_{1}\cap G_{2}$, 所以概率为 $1/2$.
        \end{itemize}
    \end{jieda}
	\pause
	\begin{rmk}
	假设一个强制性法律规定:如果一个男孩有姐妹则禁止他走出家门。那么这时 ``随机遇到的孩子是女孩" 就等价于 `` 至少有一个孩子是女孩”
	\end{rmk}

\end{frame}

%
\begin{frame}{男孩女孩概率问题续:冬天出生的女孩}
\begin{exam}
	某家庭有两个孩子,给定条件至少一个是女孩且在冬天出生,求两个孩子都是女孩的概率。假设四个季节出生的可能性相同且性别和季节是相互独立的.
\end{exam}
\pause

    \begin{jieda}
        由条件概率的定义,可得:
        \begin{align*}
            \ P (\mbox{两个都是女孩 | 至少有一个是冬天出生的女孩})\\
            =\frac{P (\mbox{两个都是女孩,至少有一个是冬天出生的女孩})}{P (\mbox{至少有一个是冬天出生的女孩})}
        \end{align*}\pause
		由于指定的孩子是在冬天出生的女孩的概率为 $1/8$, 所以,$$P (\mbox{至少有一个是在冬天出生的女孩}) = 1 - (7/8)^2.$$
    \end{jieda}
\end{frame}


		\begin{frame}{男孩女孩概率问题续:冬天出生的女孩}
			利用性别和季节是相互独立的假设,得到:
        \begin{align*}
            &P (\mbox{两个都是女孩,至少有一个是冬天出生的女孩})\\
            &=P (\mbox{两个都是女孩,至少有一个是冬天出生的})\\
            &=(1/4) P (\mbox{至少有一个是冬天出生的女孩})\\
            &=(1/4) (1-P (\mbox{所有孩子都不是在冬天出生的})
        \end{align*}
        合在一起得到,$$P (\mbox{两个都是女孩 | 至少有一个是冬天出生的女孩})=7/15$$
\end{frame}


\begin{frame}{结绳问题}
	\vspace{-0.15cm}
	\begin{exam}\
		$n$ 根绳 $2n$ 个头两两相接,求事件 $A=\{\mbox{恰好结成} n\mbox{个圈}\}$ 的概率.
	\end{exam}
	\pause
	\vspace{-0.4cm}
	\begin{itemize}[<+-|alert@+>]
		\item 设想 $2n$ 个头排成一行,规定将第 $2k-1$ 个头与第 $2k$ 个端头相接;
		\item 令 $B_i$ 表示第 $i$ 根绳的头与尾恰好相接,则 $A=B_1B_2\cdots B_n$;
		\item 若以 $n (A)$ 表示事件 $A$ 所包含的样本点个数,则易知 \pause
		\begin{eqnarray*}
			n(\Omega)=(2n)!, \pause n(B_1)=2n(2n-2)!\pause \Rightarrow P(B_1)=\dfrac{n(B_1)}{n(\Omega)}=\dfrac{1}{2n-1};\pause
		\end{eqnarray*}
		\item $P (B_2|B_1)$ 可看作 $n-1$ 根绳某根绳头尾相接的概率,类比 $n$ 根绳情形可得 $P (B_2|B_1)=\dfrac{1}{2 (n-1)-1}=\dfrac{1}{2n-3};$
		\item 同理可得 $P (B_k|B_1B_2\cdots B_{k-1})=\dfrac{1}{2[n-(k-1)]-1}=\dfrac{1}{2n-2k+1}, 3\leq k\leq n;$
		\item 利用乘法公式可得
		\begin{align*}
			P(A)&=\pause P(B_1B_2\cdots B_n)=\pause P(B_1)P(B_2|B_1)\cdots P(B_n|B_1\cdots B_{n-1})\pause\\
			&=\pause\dfrac{1}{(2n-1)!!}
		\end{align*}
	\end{itemize}


\end{frame}

\begin{frame}{取球问题}
	\begin{exam}
		在计算机中输入程序,让它自动完成如下操作:
		\begin{itemize}[<+-|alert@+>]
			\item 在 $1-\dfrac{1}{2^n}$ 时刻,往盒中放入标号 $10 (n-1)+1\sim 10n$ 的 $10$ 个球,同时取出标号为 $10 (n-1)+1$ 的球,$n\geq 1$;
			\item 在 $1-\dfrac{1}{2^n}$ 时刻,往盒中放入标号 $10 (n-1)+1\sim 10n$ 的 $10$ 个球,同时取出标号为 $n$ 的球,$n\geq 1$;
			\item 在 $1-\dfrac{1}{2^n}$ 时刻,往盒中放入标号 $10 (n-1)+1\sim 10n$ 的 $10$ 个球,同时随机地从盒中取出一个球,$n\geq 1$.
		\end{itemize}
	\pause
		则在时刻 $1$, 盒中的球数结果如下
		\begin{itemize}[<+-|alert@+>]
			\item 盒子中有无穷多个球;
			\item 盒子变为空的;
			\item 盒子变为空的概率等于 $1$?
		\end{itemize}
	\end{exam}
\end{frame}

\begin{frame}{取球问题}
	\jieda\
	\begin{itemize}[<+-|alert@+>]
		\item 记 $E$=\{在时刻 $1$ 时盒子变空 \};
		\item $\overline{E}$=\{在时刻 $1$ 时盒中有球未被取出 \};
		\item 记 $A_k$=\{在时刻 $1$ 时 $k$ 号球仍在盒中未被取出 \}, 则 \pause$\overline{E}=\bigcup\limits_{k=1}^{\infty} A_k$;\pause
		\item 由概率的次可加性知
		$$ P( \overline{E})= P\big(\bigcup_{k=1}^{\infty}A_k\big)\leq\sum_{k=1}^{\infty} P(A_k);$$
		\item 为证 $ P (\overline{E})=0$,只需证明 $P (A_k)=0,\,k\geq 1$;
		\item 由于证法类似,仅以证明 $ P (A_1)=0$ 为例;
	\end{itemize} %,则 $ \overline{E}$=\{在时刻 $1$ 时盒中有球未被取出 \}。为证 $ P (E)=1$,只需证明 $ P ( \overline{E})=0$。


\end{frame}

\begin{frame}{取球问题}
	\begin{itemize}[<+-|alert@+>]
	\item 记 $B_n$=\{在 $1-\dfrac{1}{2^n}$ 时刻 $1$ 号球未被取出 \},\pause 易知 $A_1=\bigcap\limits_{n=1}^{\infty} B_n$;\pause
	\item 令 $C_m=\bigcap\limits_{n=1}^{m} B_n$, 则有 \pause
	\[C_m=\bigcap\limits_{n=1}^{m} B_n\supset\bigcap\limits_{n=1}^{m+1} B_n=C_{m+1},\mbox{ 且 } \lim_{m\rightarrow\infty} C_m=A_1;\]\pause
	\item 由概率的上连续性知 $$ P (A_1)= P\big (\lim_{m\rightarrow\infty} C_m\big)=\lim_{m\rightarrow\infty} P\big (C_m\big);$$
	\item 下面求 $P (C_m)$, 即 $P\big (\bigcap_{n=1}^{m} B_n\big)=\prod_{n=1}^{m}\frac{9n}{9n+1}=\prod_{n=1}^{m}\left (1-\frac{1}{9n+1}\right)$;
	\begin{itemize}[<+-|alert@+>]
		\item $B_1$: 在 $1/2$ 时刻,盒中 $10$ 个球,$1$ 号球未被取出,故 \pause $ P (B_1)=\frac{9}{10}$;\pause
		\item $B_2$: 在 $3/4$ 时刻,盒中 $19$ 个球,$1$ 号球未被取出,故 \pause $ P (B_2|B_1)=\frac{18}{19}$;\pause
		\item $B_n$: $ P(B_n|B_1B_2\cdots B_{n-1})=\frac{9n}{9n+1}.$
	\end{itemize}
\item $ P (A_1)=\prod_{n=1}^{\infty}\big (1-\frac{1}{9n+1}\big)=0.$ 同理可证 $ P (A_k)=0,\,k=2,3,\cdots$.
\end{itemize} %,则 $ \overline{E}$=\{在时刻 $1$ 时盒中有球未被取出 \}。为证 $ P (E)=1$,只需证明 $ P ( \overline{E})=0$。



\end{frame}

%\begin{frame}{取球问题}
%	在时刻 $\dfrac{3}{4}$ 时,盒中共有 $19$ 个球,若 $1$ 号球仍未被取出,则在 $B_1$ 发生的条件下还有 $B_2$ 发生,此时有 $18$ 种取球方式,由于面对的是变化了的概率空间,故按无条件概率的求法计算出 $$ P (B_2|B_1)=\frac{18}{19}.$$
%	一般地,有 $$ P (B_n|B_1B_2\cdots B_{n-1})=\frac{9n}{9n+1}.$$
%	于是按照概率的乘法定理得到 $$ P\left (\bigcap_{n=1}^{m} B_n\right)=\prod_{n=1}^{m}\frac{9n}{9n+1}=\prod_{n=1}^{m}\left (1-\frac{1}{9n+1}\right).$$
%\end{frame}
%
%\begin{frame}{取球问题}
%	于是就有 $$ P (A_1)=\lim_{m\rightarrow\infty} P\left (\bigcap_{n=1}^{m} B_n\right)=\prod_{n=1}^{\infty}\left (1-\frac{1}{9n+1}\right).$$
%	由于 $$\sum_{n=1}^{\infty}\frac{1}{9n+1}=\infty,$$ 所以由无穷乘积发散的判别准则,知 $$ P (A_1)=\prod_{n=1}^{\infty}\left (1-\frac{1}{9n+1}\right)=0.$$
%	同理可证 $ P (A_k)=0,\,k=2,3,\cdots$。综合上述,就证出了在时刻 $1$ 时,盒子变为空的概率等于 $1$。
%\end{frame}
%
%\begin{frame}
%	\begin{itemize}
%		\item 上一题的解答需要清晰正确的转换思路:$E\rightarrow  \overline{E}\rightarrow A_1\rightarrow\bigcap\limits_{n=1}^{m} B_n$
%		\item 概率论中的许多问题都可以用罐中取球的模型来描述
%		\item 下一例出现的罐子模型是所谓有后效的模型,可用来粗略描述流行病的传播规律
%	\end{itemize}
%\end{frame}








\begin{frame}
	\frametitle{罐子模型(波利亚模型)}
	\begin{exam}
		设罐中有 $b$ 个黑球,$r$ 个红球,每次随机的取出一球,取出后将原球放回,还加进 $c$ 个同色球和 $d$ 个异色球。记 $B_i:=\{\mbox{第} i\mbox{次取出的是黑球}\}, R_j:=\{\mbox{第} j\mbox{次取出的是红球}\}$. 若连续从罐子中取出三个球,其中有两个红球,一个黑球,则由乘法公式可得
		\begin{eqnarray*}
			P(B_1R_2R_3)&=&P(B_1)P(R_2|B_1)P(R_3|B_1R_2)\\
			&=&\frac{b}{b+r}\cdot\frac{r+d}{b+r+c+d}\cdot\frac{r+d+c}{b+r+2c+2d}\\
			P(R_1B_2R_3)&=&P(R_1)P(B_2|R_1)P(R_3|R_1B_2)\\
			&=&\frac{r}{b+r}\cdot\frac{b+d}{b+r+c+d}\cdot\frac{r+d+c}{b+r+2c+2d}\\
			P(R_1R_2B_3)&=&P(R_1)P(R_2|R_1)P(B_3|R_1R_2)\\
			&=&\frac{r}{b+r}\cdot\frac{r+c}{b+r+c+d}\cdot\frac{b+2d}{b+r+2c+2d}
		\end{eqnarray*}
		显然以上概率与黑球在第几次抽取有关.
	\end{exam}
\end{frame}
\begin{frame}
	\frametitle{罐子模型(波利亚模型)}
	\vspace{-0.3cm}
	\begin{itemize}[<+-|alert@+>]
		\item 当 $c=-1, d=0$ 时,即为不返回抽样。此时前次抽取结果会影响后次抽取结果,但只要抽取的黑球与红球个数确定,则概率不依赖其抽出球的次序,都是一样的.
		{\small\begin{eqnarray*}
				P(B_1R_2R_3)= P(R_1B_2R_3)=P(R_1R_2B_3)=\frac{br(r-1)}{(b+r)(b+r-1)(b+r-2)}.
		\end{eqnarray*}}
		\item 当 $c=0,d=0$ 时,即为返回抽样。此时前次抽取结果不会影响后次抽取结果,故上述三个概率相等且都等于
		{\small\begin{eqnarray*}
				P(B_1R_2R_3)= P(R_1B_2R_3)=P(R_1R_2B_3)=\frac{br^2}{(b+r)^3}.
		\end{eqnarray*}}
		\item 当 $c>0,d=0$ 时,称为传染病模型。此时每次取出球后会增加下一次取出同色球的概率,或换言之,每发现一个传染病患者,以后都会增加再传染的概率。与前两种情况一样,三个概率都等于
		{\small\begin{eqnarray*}
				P(B_1R_2R_3)= P(R_1B_2R_3)=P(R_1R_2B_3)=\frac{br(r+c)}{(b+r)(b+r+c)(b+r+2c)}.                                                        \end{eqnarray*}}

	\end{itemize}
\end{frame}
\begin{frame}
	\frametitle{罐子模型(波利亚模型)}
	\begin{itemize}[<+-|alert@+>]
		\item 从上面的结果可以看出,只要 $d=0$,以上三个概率都相等,即只要抽取的黑球与红球的个数确定,则概率不依赖于抽出黑红球的次序.
		\item 当 $c=0,d>0$ 时,称为安全模型。此模型可解释为:每当事故发生了 (当红球被取出),安全工作就抓紧一些,下次再发生事故的概率就会减少,而当事故没有发生时 (黑球被取出),安全工作就放松一些,下次再发生事故的概率就会增大,此时,上述三个概率分别为
		{\small\begin{eqnarray*}
				P(B_1R_2R_3) &=&\frac{b}{b+r}\cdot\frac{r+d}{b+r+d}\cdot\frac{r+d}{b+r+2d}\\
				P(R_1B_2R_3)&=&\frac{r}{b+r}\cdot\frac{b+d}{b+r+d}\cdot\frac{r+d}{b+r+2d}\\
				P(R_1R_2B_3) &=&\frac{r}{b+r}\cdot\frac{r}{b+r+d}\cdot\frac{b+2d}{b+r+2d}
		\end{eqnarray*}}
	\end{itemize}

\end{frame}

\begin{frame}{罐子模型(波利亚模型)}
	\begin{exam}
	设罐中有 $b$ 个黑球,$r$ 个红球,每次随机取出一球后将原球放回并加进 $c$ 个同色球,如此反复进行。试证明:在前 $n=n_1+n_2$ 次取球中,取出了 $n_1$ 个红球和 $n_2$ 个黑球的概率为
		$$C_n^{n_1}\frac{a(a+c)(a+2c)\cdots(a+n_1c-c)b(b+c)(b+2c)\cdots(b+n_2c-c)}{(a+b)(a+b+c)(a+b+2c)\cdots(a+b+nc-c)}.$$
	\end{exam}
%	\begin{jieda}
%		记 $A_k$=\{第 $k$ 次取球时取出白球 \}, 于是 $A_k^c$=\{第 $k$ 次取球时取出黑球 \}. 采用逐个考虑被改变了的概率空间的方法,不难利用乘法定理求得
%		\begin{align*}
%			& P(A_1\cdots A_{n_1}A_{n_1+1}^c\cdots A_n^c)\\=&\frac{a(a+c)(a+2c)\cdots(a+n_1c-c)b(b+c)(b+2c)\cdots(b+n_2c-c)}{(a+b)(a+b+c)(a+b+2c)\cdots(a+b+nc-c)}.
%		\end{align*}
%	\end{jieda}
\end{frame}
\subsection{事件的独立性}

\begin{frame}
  \frametitle{两个事件的独立性}
  \begin{itemize}[<+-|alert@+>]
  \item 直观上来说,两个事件的独立性是指:一个事件的发生不影响另一个事件的发生。比如在掷两颗骰子的实验中,第一颗骰子的点数和第二颗骰子的点数是互不影响的.
  \item 从概率的角度看,$P (A|B)$ 与 $P (A)$ 的差别在于:事件 $B$ 的发生改变了事件 $A$ 发生的概率,也即事件 $B$ 对事件 $A$ 有某种影响。故如果 $A$ 与 $B$ 的发生是相互不影响的,则有 \pause
    \begin{eqnarray*}
      P(A|B)=P(A),\quad P(B|A)=P(B).
    \end{eqnarray*}
    \pause 上面两式均等价于
    \begin{eqnarray}\label{eq:inden}
      P(AB)=P(A)P(B)
    \end{eqnarray}
  \item 注意到 \eqref{eq:inden} 式对 $P (B)=0$ 或 $P (A)=0$ 仍然成立,为此,我们用 \eqref{eq:inden} 作为两个事件相互独立的定义.
  \end{itemize}
\end{frame}

\begin{frame}
  \frametitle{两个事件独立性的定义}
  \begin{defi}
    如果对事件 $A$ 与 $B$ 有
    \begin{eqnarray*}
      P(AB)=P(A)P(B)
    \end{eqnarray*}
    成立,则称 \textcolor{red}{事件 $A$ 与 $B$ 相互独立}, 简称 \textcolor{red}{$A$ 与 $B$ 独立}. 否则称 $A$ 与 $B$ 不独立或相依.
  \end{defi}

\begin{rmk}
	\begin{itemize}
		\item 零概率事件 $E$ 与任何事件相互独立,特别的,不可能事件与任何事件相互独立
        \item 非零概率互不相容事件,一定不独立
        \item 非零概率相互独立事件,一定相容
\end{itemize}


\end{rmk}

\pause
  \textcolor{red}{如何确定事件的独立性: }
  \begin{itemize}[<+-|alert@+>]
  \item 实际问题中,两个事件的独立大多根据经验及相互有无影响的直观性来判断.
  \item 但对于较复杂事件,有无相互影响并不是很直观,则需要验证 \eqref{eq:inden} 式是否成立来说明独立性.
  \end{itemize}
\end{frame}


 \begin{frame}
	\frametitle{对立事件的独立性}
	\begin{thm}
		若 $A$ 与 $B$ 独立,则 $A$ 与 $\overline{B}$ 独立,$\overline{A}$ 与 $B$ 独立,$\overline{A}$ 与 $\overline{B}$ 独立.
	\end{thm}
	\pause

	\zheng 我们仅证 $P (A\overline{B})=P (A) P (\overline{B})$, 其余类似可证.
	\begin{eqnarray*}
		P(A\overline{B})&=&\pause P(A-B)=\pause P(A)-P(AB)\pause =P(A)-P(A)P(B)\\
		&=&\pause P(A)(1-P(B))=\pause P(A)P(\overline{B})
	\end{eqnarray*}

	\pause
	对于上面的定理直观上来理解也是很容易的:因 $A,B$ 独立,故 $A$ 的发生不影响 $B$ 的发生,从而也不会影响 $B$ 的不发生,$\cdots$
\end{frame}









   \begin{frame}
	\frametitle{三个事件的独立性}
	\begin{defi}
		设 $A,B,C$ 三个事件,如果有
		\begin{eqnarray}\label{eq:inden1}
			\left.\begin{array}{l}
				P(AB)=P(A)P(B)\\
				P(AC)=P(A)P(C) \\
				P(BC)=P(B)P(C)
			\end{array}\right\}\\
			\label{eq:inden2}
			P(ABC)=P(A)P(B)P(C)
		\end{eqnarray}
		则称 $A,B,C$ 相互独立。如果仅有 \eqref{eq:inden1} 式成立,则称 $A,B,C$ 两两独立.
	\end{defi}
\end{frame}

\begin{frame}
	\frametitle{两两独立与相互独立的关系}
	\begin{itemize}[<+-|alert@+>]
		\item 由定义可知,三个事件相互独立必能推出两两独立.
		\item 但两两独立未必能推出相互独立,即 \eqref{eq:inden1} 式成立,不一定能推出 \eqref{eq:inden2} 成立
		\begin{itemize}
			\item 考虑独立投掷两枚均匀硬币的随机试验,设事件 $A$ 代表第一枚硬币正面朝上,事件 $B$ 代表第二枚硬币正面朝上,事件 $C$ 表示两枚硬币结果相同。易知: \pause
			$A$ $B$ 和 $C$ 是两两独立,但
			\begin{align*}
				P(A\cap B\cap C)=1/4\neq 1/8=P(A)P(B)P(C).
			\end{align*}
			\item 考虑一个均匀的正四面体,第一二三面分别染上红 / 白 / 黑色,第四面同时染上红白黑色。现在以 $A,B,C$ 分别记投一次四面体出现红,白,黑色朝下的事件。则易有 \pause
			\begin{eqnarray*}
				P(A)=P(B)=P(C)=\pause 1/2\\ \pause
				P(AB)=P(BC)=P(AC)=\pause 1/4\\ \pause
				P(ABC)=\pause 1/4    \pause
			\end{eqnarray*}
		\end{itemize}\vspace{-0.7cm}
	\end{itemize}
\end{frame}

		\begin{frame}
			\frametitle{两两独立与相互独立的关系}
			\begin{itemize}[<+-|alert@+>]
		\item 反之,如果 \eqref{eq:inden2} 成立,是否能推出 \eqref{eq:inden1} 成立?
		\begin{itemize}[<+-|alert@+>]
			\item 考虑一个均匀的正八面体,第 1, 2, 3, 4 面染上红色,第 1, 2, 3, 5 面染上白色,第 1, 6, 7, 8 面染上黑色。现在以 $A,B,C$ 分别记投一次八面体出现红,白,黑色朝下的事件,则 \pause
			\begin{eqnarray*}
				P(A)=P(B)=P(C)=\pause 4/8=1/2\\ \pause
				P(ABC)=\pause 1/8   \\ \pause
				P(AB)=3/8\pause \neq 1/4=P(A)P(B)
			\end{eqnarray*}
		\end{itemize}
	\end{itemize}
\end{frame}

\begin{frame}
	\frametitle{三个以上事件的独立性}
	\begin{defi}
		设 $(\Omega,\mathcal{F}, P)$ 为一概率空间,$A_1,A_2,\cdots,A_n\in\mathcal{F}$, 对任意的 $1\le k\le n$ 及任意的 $1< j_1<j_2<\cdots<j_k\leq n$ 均有:
		\begin{eqnarray}\label{eq:mulinden0}
			P(A_{j_1}A_{j_2}\cdots A_{j_k})=P(A_{j_1})P(A_{j_2})\cdots P(A_{j_k})
		\end{eqnarray}
		成立,则称事件 $A_1,\cdots, A_n$ 相互独立.
	\end{defi}
	\pause
	\begin{itemize}[<+-|alert@+>]
		\item \eqref{eq:mulinden0} 式共有多少个等式?\pause
		\begin{eqnarray}
			\label{eq:mulinden}
			\left.\begin{array}{l}
				P(A_{j_1}A_{j_2})=P(A_{j_1})P(A_{j_2})\\
				P(A_{j_1}A_{j_2}A_{j_3})=P(A_{j_1})P(A_{j_2})P(A_{j_3}) \\
				\qquad \vdots\\
				P(A_1A_2\cdots A_n)=P(A_1)P(A_2)\cdots P(A_n)
			\end{array}\right\} \pause \textcolor{red}{C_n^2+\cdots+C_n^n=2^n-n-1}
		\end{eqnarray}
		\pause
		\item 从定义可以看出,$n$ 个相互独立事件中的任取 $m$($2\le m\le n$) 个事件仍是相互独立的,而且任意一部分与另一部分也是独立的.
		\item 类似于前面的证明,将相互独立事件中的任一部分换为对立事件,所得诸事件仍是相互独立的.
	\end{itemize}



\end{frame}

\begin{frame}
	\frametitle{任意多个事件相互独立}
	\begin{defi}
		设 $(\Omega,\mathcal{F}, P)$ 为一概率空间,每个 $t\in T$ 有 $A_t\in \mathcal{F}$. 称 $\{A_t, t\in T\}$ 为独立事件族,如果对 $T$ 的任意有限子集 $\{t_1,t_2,\cdots, t_s\}$, 事件 $A_{t_1}, A_{t_2},$ $\cdots, A_{t_s}$ 相互独立.
	\end{defi}


	\vspace{0.8cm}
	\pause
	\begin{exam}
		$\mathcal{F}$ 中事件序列 $\{A_n\}$ 为相互独立的充分必要条件是,任意 $n\geq 1$, 事件 $A_1,A_2, \cdots, A_n$ 独立;等价的,任意有限个自然数 $k_1,\cdots, k_s$ 有
		\begin{eqnarray*}
			P(A_{k_1}A_{k_2}\cdots A_{k_s})=P(A_{k_1})P(A_{k_2})\cdots P(A_{k_s})
		\end{eqnarray*}

	\end{exam}

\end{frame}


\begin{frame}{条件独立}
\begin{defi}
称事件 $A$ 和 $B$ 是关于 $E$ 条件独立的,如果 $$P (A\cap B|E)=P (A|E) P (B|E)$$
\end{defi}
\pause
\begin{itemize}[<+-|alert@+>]
\item 两个事件可以在给定事件 $E$ 的条件下是条件独立的,但它们不是独立的.
\item 两个事件可以是独立但却不是关于 $E$ 条件独立的.
\item 两个事件可以关于 $E$ 条件独立但关于 $E^c$ 不存在条件独立.
\end{itemize}
\end{frame}

\begin{frame}{条件独立不意味着独立}
\begin{exam}
假设有两枚硬币,一枚是均匀的,一枚是不均匀的。从两枚硬币中随机的选一枚硬币并进行抛掷 2 次,若令
\begin{align*}
  F &:=\{\mbox{选取的硬币是均匀的}\}\\
   A_{1}&:= \{\mbox{第一次投掷硬币正面朝上}\}\\
   A_{2}&:= \{\mbox{第二次投掷硬币正面朝上}\}
\end{align*}
则给定 $F$ 为条件,$A_1$ 和 $A_2$, 是相互独立的,$A_1$ 和 $A_2$ 并不是无条件独立的,因为 $A_1$ 会提供关于 $A_2$ 的信息.
\end{exam}
\end{frame}

\begin{frame}{独立不意味着条件独立}
\begin{exam}
假设只有我的朋友 Alice 和 Bob 给我打过电话。每天他俩都会相互独立地决定是否给我打电话。若令
\begin{align*}
	 A&:= \{\mbox{Alice 给我打电话}\}\\
	 B&:= \{\mbox{Bob 给我打电话}\}\\
     R&:=\{\mbox{听到电话铃响}\}
  \end{align*}
  \pause
  \begin{itemize}[<+-|alert@+>]
  \item 显然,$A$ 和 $B$ 是无条件独立的.
  \item 但现在我听到一声电话铃响,那 $A$ 和 $B$ 就不再独立了:如果这个电话不是 Alice 打的,那就肯定是 Bob 打的。从而 \pause
  $$P(B|R)<1=P(B|A^cR)= \frac{P(BA^cR)}{P(A^cR)}= \frac{P(BA^c|R)}{P(A^c|R)}.$$
  显然: $P (BA^c|R)>P (B|R) P (A^c|R)$
  \item $B$ 与 $A^c$ 关于 $R$ 不条件独立,$A,B$ 亦是如此.
  \end{itemize}

\end{exam}
\end{frame}

\begin{frame}{给定 $E$ 条件独立 $vs$ 给定 $E^c$ 条件独立}
\begin{exam}
\label{27}
假设有两种课程:好的课程和坏的课程。在好的课上,如果你努力,就很有可能得到 $A$. 在坏的课上,教授随机分配给学生分数,而不管他们是否努力。若令
\begin{align*}
	G&:= \{\mbox{这个课程是好的}\}\\
	W&:= \{\mbox{你学习努力}\}\\
	A&:=\{\mbox{你的得分为} A\}
 \end{align*}
 \pause 这时,给定 $G^c$, $A$ 和 $W$ 是条件独立的,但给定 $G$, $A$ 和 $W$ 却不是独立的!

\end{exam}
\end{frame}

\subsection{随机试验的独立性}
 \begin{frame}
              \frametitle{随机试验的独立性}

               \begin{itemize}[<+-|alert@+>]
               \item 先考虑两个随机试验,假定 $(\Omega_i,\mathcal{F}_i,P_i), i=1,2$ 为第 $i$ 个随机试验对应的概率空间。按照之前独立性的理解,两个试验的独立性应当叙述为:\pause
                \textcolor{red}{ \begin{eqnarray*}
                   &&\mbox{对任何的} A_i\in\mathcal{F}_i, i=1,2, A_1\mbox{与} A_2\mbox{同时}\\
                   &&\mbox{发生的概率等于它们各自概率之乘积}
                 \end{eqnarray*}}
             \item 两个不妥:
               \begin{itemize}[<+-|alert@+>]
               \item ``$A_1$ 与 $A_2$ 同时发生" 应当是这两个事件的交,但它们分别是两个样本空间 $\Omega_1,\Omega_2$ 的子集,无法进行运算;
               \item 两个概率空间有各自的概率 $P_1, P_2$, 但涉及两个度验,命题中 ``同时发生的概率" 既不能用 $P_1$ 也不能用 $P_2$ 来度量.
               \end{itemize}
             \item 解决方法:构造可以同时描述两个试验的新概率空间 $(\Omega,\mathcal{F},P)$.
               \end{itemize}
             \end{frame}

             \begin{frame}
               \frametitle{乘积空间的构造}
               \begin{itemize}[<+-|alert@+>]
               \item 样本乘积空间: $\Omega:=\Omega_1\times \Omega_2=\{(\omega_1,\omega_2):\omega_1\in\Omega_1\mbox{且}\omega_2\in \Omega_2\}$;
               \item 乘积 $\sigma$- 代数 $\mathcal{F}_1\times\mathcal{F}_2$:
                 \begin{itemize}[<+-|alert@+>]
                 \item 可测矩形集类: $\mathcal{G}:=\{A_1\times A_2: A_1\in\mathcal{F}_1, A_2\in \mathcal{F}_2\}$;
                 \item $\mathcal{F}_1\times \mathcal{F}_2:=\sigma(\mathcal{G})$
                 \end{itemize}
               \item 乘积概率测度:
                 \begin{itemize}[<+-|alert@+>]
                 \item 对于每个可测矩形 $A_1\times A_2\in \mathcal{G}$ 定义如下集函数:
                   \begin{eqnarray}\label{eq:timeprob}
                     P(A_1\times A_2)=P_1(A_1)P_2(A_2), \quad A_i\in\mathcal{F}_i, i=1,2.
                   \end{eqnarray}
                 \item 理论上可以证明如上定义在 $\mathcal{G}$ 上的集函数 $P$ 可唯一地扩张为 $\mathcal{F}_1\times\mathcal{F}_2$ 上的概率测度,称之为 $P_1$ 与 $P_2$ 的乘积 (概率) 测度.
                 \end{itemize}
               \item 在上述乘积测度下
                 \begin{eqnarray*}
                   &&P(A_1\times \Omega_2)=P_1(A_1), \quad P(\Omega_1\times A_2)=P_2(A_2)\\\pause
                   &&\pause P\big((A_1\times \Omega_2)\cap (\Omega_1\times A_2)\big)=\pause P(A_1\times A_2)\\
                   &&\pause = P_1(A_1)P_2(A_2)=\pause P(A_1\times\Omega_1)P(\Omega_1\times A_2)
                 \end{eqnarray*}
               \item $(\Omega_i,\mathcal{F}_i,P_i)$ 的独立性取决于乘积样本空间 $\Omega_1\times\Omega_2$ 上的概率是否取作由 (\ref{eq:timeprob}) 所确定的乘积测度
               \end{itemize}
             \end{frame}

             \begin{frame}
               \frametitle{$n$ 个试验相互独立的定义}

               \begin{defi}
                 设有 $n$ 个随机试验,第 $i$ 个试验的概率空间为 $(\Omega_i,\mathcal{F}_i,P_i),$ $ i=1,\cdots,n$. 代表这 $n$ 个试验的乘积样本空间 $\Omega=\Omega_1\times \cdots \times \Omega_n$, $\mathcal{F}=\mathcal{F}_1\times \cdots\times \mathcal{F}_n=\sigma (\mathcal{G})$, 其中 $\mathcal{G}$ 为形如 $B_1\times\cdots\times B_n (B_i\in\mathcal{F}_i)$ 的可测矩形的全体。如果 $(\Omega,\mathcal{F})$ 上的概率测度 $P$ 是 $P_1,\cdots, P_n$ 的乘积测度,即对任何 $B_1\times\cdots\times B_n\in \mathcal{G}$ 满足
                 \begin{eqnarray*}
                   P(B_1\times\cdots\times B_n)=P_1(B_1)\cdots P(B_n),
                 \end{eqnarray*}
                 则称这 $n$ 个度验独立. \pause 如果现设 $$\Omega_i\equiv \Omega_0, \mathcal{F}_i\equiv \mathcal{F}_0, P_i\equiv P_0, i=1,\cdots,n, $$ 即 $n$ 个试验有相同的概率空间,则称它们为 $n$ 个 (重) 独立重复试验. \pause 如果在 $n$ 个独立重复实验中,每次试验的可能结果为两个:$A$ 或 $\overline{A}$, 则称这种试验为 \textcolor{red}{$n$ 重伯努利试验}.
               \end{defi}
             \end{frame}
                          \begin{frame}{彩票问题}
               % \frametitle{试验的独立性}
               % \begin{defi}
               %   设有两个实验 $E_1$ 和 $E_2$,假如实验 $E_1$ 的任一结果(事件)与试验 $E_2$ 的任一结果(事件)都是相互独立事件,则称这两个实验相互独立.
               % \end{defi}
               % \pause
               % \begin{defi}
               %   如果 $n$ 个实验 $E_1,E_2,\cdots,E_n$ 的任一结果都是相互独立的事件,则称试验 $E_1,\cdots E_n$ 相互独立。如果这 $n$ 个实验是相同的,则称其为 \textcolor{red}{$n$ 重独立重复实验}. 如果在 $n$ 重独立重复实验中,每次试验的可能结果为两个:$A$ 或 $\overline{A}$, 则称这种试验为 \textcolor{red}{$n$ 重伯努利试验}.
               % \end{defi}
               % \pause

               \begin{exam}
                 某彩票每周开奖一次,每次提供十万分之一的中奖机会,且各周开奖是独立的。若你每周买一张彩票,坚持十年(每年按 52 周计算),试求未中奖的概率.
               \end{exam}

               \pause  \jieda 依假设,每次中奖的概率为 $10^{-5}$, 于是每次不中奖的概率是 $1-10^{-5}$. 另外十年一共购买 520 次彩票,而每次开奖都是独立的,相当于进行了 520 次独立重复试验. \pause 若记 $A_i$ 为 “第 $i$ 次开奖不中奖”, 则 $A_1,\cdots, A_{520}$ 相互独立,从而
               \begin{eqnarray*}
                 P(A_1A_2\cdots A_{520})=(1-10^{-5})^{520}=0.9948
               \end{eqnarray*}

             \end{frame}




%%% Local Variables:
%%% mode: latex
%%% TeX-master: t
%%% End:
