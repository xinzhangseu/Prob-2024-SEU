\setcounter{section}{1}
\section{随机变量}
\title [概率论]{第九讲:随机变量定义及相关性质}
\author [张鑫 {\rm Email: xzhangseu@seu.edu.cn} ]{\large 张 鑫}
\institute [东南大学数学学院]{\large \textrm{Email: xzhangseu@seu.edu.cn} \\ \quad  \\
	\large 东南大学 \quad 数学学院 \\
	\vspace{0.3cm}
	%  \trc{公共邮箱: \textrm{zy.prob@qq.com}\\
	%    \hspace{-1.7cm}  密 \qquad 码: \textrm{seu!prob}}
}
\date{}



{ \setbeamertemplate{footline}{}
	\begin{frame}
		\titlepage
	\end{frame}
}

% \begin{frame}[plain]
%   \frametitle{目录}
%   \setcounter{tocdepth}{2}
%   \tableofcontents
% \end{frame}
\addtocounter{framenumber}{-3}  % 目录页不计算页码
%\section{随机变量}

\subsection{随机变量的定义及性质}


\begin{frame}{随机变量的引入}
	\begin{itemize}[<+-|alert@+>]
		\item 赌徒输光:甲和乙初始资金分别为 $i, a-i$ 元,每一局甲赢的概率为 $p$%\in (0,1)$.
		\item 关注的问题
		      \begin{itemize}[<+-|alert@+>]
			      \item 甲最终获胜的概率
			      \item 甲乙两人在任意时刻的剩余资产:$k$ 轮赌博后恰好剩下 $j$ 元
			      \item $k$ 轮赌博后甲乙两人资产的差额 $Z$
			      \item 赌博持续时间 $R$
		      \end{itemize}
		\item 表示方法:
		      \begin{itemize}[<+-|alert@+>]
			      \item $E:=$\{甲最终获胜 \}, $Q_i:=P$(E)
			      \item $A_{jk}:=$\{甲在 $k$ 轮赌博后恰好剩下 $j$ 元 \}
			      \item $B_{jk}:=$\{乙在 $k$ 轮赌博后恰好剩下 $j$ 元 \}
			      \item $k$ 轮赌博后甲乙两人资产的差额如何表示?
			      \item 赌博持续时间 $R$ 如何表示?
			      \item 很难用事件来表示或者表示很复杂
		      \end{itemize}

	\end{itemize}

\end{frame}

\begin{frame}{随机变量的引入}
	\begin{itemize}[<+-|alert@+>]
		\item $X_k:=$ 甲在 $k$ 轮赌博后的资产
		      \begin{itemize}[<+-|alert@+>]
			      \item 乙在 $k$ 轮赌博后的资产 $Y_k=a-X_k$
			      \item 资产差额:$Z=X_k-Y_k=2X_k-a$
			      \item 持续时间:$R=\min\{n: X_n=0, \mbox{或} Y_n=0\}$
		      \end{itemize}
		\item $X_k$ 取值的特点
		      \begin{itemize}[<+-|alert@+>]
			      \item 依赖于前面 $k$ 次赌博这一 “随机试验” 的结果
			      \item 在 “随机试验” 完成之前,$X_k$ 取值不确定,因此具有不确定性
			      \item $k$ 次赌博 “随机试验” 一旦完成,$X_k$ 的值必然确定
			      \item 记 $k$ 次赌博 “随机试验” 样本空间为 $\Omega$, 则给定 $\omega\in\Omega$, 则 $X_k$ 值确定
		      \end{itemize}
		\item 以 $k=2$ 为例,看一下 $X_2$ 的取值情况,设 $i\geq 2$
		      \begin{itemize}[<+-|alert@+>]
			      \item $\Omega=\{\omega_1,\omega_2, \omega_3,\omega_4\}$, 其中 $\omega_1=(\mbox{胜,胜}), \omega_2=(\mbox{胜,败}), \omega_3=(\mbox{败,胜}), \omega_4=(\mbox{败,败})$
			      \item $X_2(\omega_1)=i+2,\  X_2(\omega_2)=X_2(\omega_3)=i,\  X_2(\omega_4)=i-2$
		      \end{itemize}
		\item $X_k$ 可以看作定义在样本空间 $\Omega$ 上的函数,即 $X_k:\Omega\rightarrow \{0, 1,\cdots, a\}$
		\item 一般的,随机变量可以看作从样本空间到实数的映射:$X:\Omega\rightarrow R$

	\end{itemize}

\end{frame}
\begin{frame}{随机变量的直观定义}
	\begin{defi} \textcolor{cyan}{(直观定义)}
		称 $X$ 为随机变量,如果 $X$ 是从样本空间 $\Omega$ 到实数的映射,即 $X:\Omega\rightarrow R$.
	\end{defi}

\end{frame}
\begin{frame}{一个随机变量的例子}
	\begin{exam}\label{312}
		考虑硬币抛掷两次的随机试验,其样本空间为 $$\Omega=\{HH,HT,TH,TT\}.$$ % 这里存在该空间上的一些随机变量 (用于实践,你也可以想出一些自己的随机变量). 每一个随机变量都是试验在某方面的数值表示.%
		\begin{itemize}[<+-|alert@+>]
			\item 令$X$表示正面朝上的次数,则$X$是一个随机变量,相应的映射如下
			      % 其可能的取值为 0、1、2. 将其看作是一个函数,$X$ 作用在 $HH$ 上的值为 2,$X$ 作用在 $TH$ 或者 $HT$ 上的值为 1,$X$ 作用在 $TT$ 上的值为 0, 即
			      $$X(TT)=0, X(HT)=X(TH)=1, X(HH)=2.$$
			\item 令 $Y$ 表示反面朝上的次数,则 $Y=2-X$, 对于任意的 $\omega\in \Omega$, 有 $Y (\omega)=2-X (\omega)$.
			\item 设 $I$ 是由第一次掷硬币的结果决定的随机变量:若第一次硬币正面朝上则 $I=1$, 反之 $I=0$, 即
			      \begin{align*}
				      I(HH)=I(HT)=1,\quad I(TH)=I(TT)=0.
			      \end{align*}
			\item 若正面记 1, 反面记 0, 此时 $\Omega=\{(1,1),(1,0),(0,1),(0,0)\}$. 上述的 $X,Y$ 和 $I$ 可表示为:$$X (\omega_1,\omega_2)=\omega_1+\omega_2,Y (\omega_1,\omega_2)=2-\omega_1-\omega_2,I (\omega_1,\omega_2)=\omega_1.$$
		\end{itemize}
	\end{exam}
\end{frame}


\begin{frame}{与函数对比分析}
	\begin{itemize}[<+-|alert@+>]
		\item 数学分析中的函数 $f (x)$
		      \begin{itemize}[<+-|alert@+>]
			      \item $f (x):D\rightarrow R$, 其中 $D\subset R$
			      \item $R$ 上可定义距离 $d (x,y)=|x-y|$
			      \item 可根据距离 $d$ 定义函数的连续性 %
		      \end{itemize}
		\item 概率论中的随机变量 $X$
		      \begin{itemize}[<+-|alert@+>]
			      \item $X(\omega):\Omega\rightarrow R$
			      \item 定义域 $\Omega$ 没有距离的定义,\pause 但有事件域 $\mathcal{F}$ 及定义其上的概率 $P$, 即具有结构 $(\Omega, \mathcal{F}, P)$\pause
			      \item 值域 $R$ 有距离,\pause 但因定义域无距离,故无法考虑随机变量的连续性 \pause
			      \item 值域 $R$ 还有 $\sigma$- 代数 $\mathcal{B}:=\mathcal{B}(R)$, 有可测空间结构 $(R,\mathcal{B})$
		      \end{itemize}
		\item 是否需要对 $X (\omega):\Omega\rightarrow R$ 做一些额外的限定,以便更好的研究 $X$?
		\item 从赌徒输光问题可以看出,对随机变量 $X$, 我们会关注
		      \begin{itemize}[<+-|alert@+>]
			      \item $X (\omega)=x$ 的概率,\pause $X (\omega)\leq x, X (\omega)\in [b,c]$ 的概率 \pause
			      \item 更一般的,$\forall B\in\mathcal{B}(R), X (\omega)\in B$ 的概率
		      \end{itemize}
		\item 概率的定义域是 $\mathcal{F}$, 要想计算 $X (\omega)\in B$ 的概率,当且仅当
		      \[\forall B\in\mathcal{B}(R), \ X^{-1}(B):=\{\omega: X(\omega)\in B\}\in\mathcal{F}\]
	\end{itemize}
\end{frame}






\begin{frame}
	\frametitle{随机变量的严格数学定义}
	\vspace{-0.1cm}
	\begin{defi}[可测映射] 设 $(\Omega,\mathcal{F})$ 和 $(E,\mathcal{E})$ 为两个可测空间并令 $X$ 为从样本空间 $\Omega$ 到 $E$ 的映射,即 $X (\omega):\Omega\rightarrow E$. 若对任意的 $B\in \mathcal{E}$ 均有
		\begin{eqnarray*}
			X^{-1}(B):=\{\omega: X(\omega)\in B\}\in \mathcal{F}
		\end{eqnarray*}
		则称 $X$ 为 $(\Omega,\mathcal{F})$ 到 $(E,\mathcal{E})$ 的可测映射.
	\end{defi}
	\pause %

	若将上述定义中的可测空间 $(E,\mathcal{E})$ 更换为 $(R,\mathcal{B})$,则 % 所定义的可测映射称为可测空间 $(\Omega,\mathcal{F})$ 上的可测函数或随机变量,即:
	\pause
	\begin{defi}[可测函数或随机变量] \label{rvdefi1}\hspace{-0.2cm} 设 $(\Omega,\mathcal{F})$ 是可测空间,$X$ 为从样本空间 $\Omega$ 到实数集 $R$ 的映射,即 $X (\omega):\Omega\rightarrow R$. 如果对 $\forall B\in \mathcal{B}$ 均有
		\[X^{-1}(B):=\{\omega: X(\omega)\in B\}\in \mathcal{F}.\]%
		则称 $X$ 为可测空间 $(\Omega,\mathcal{F})$ 上的可测函数或随机变量.
	\end{defi}%
	\pause
	\begin{defi}[随机变量的另一定义]\hspace{-0.2cm} \label{rvdefi2} 设 $(\Omega,\mathcal{F})$ 是可测空间,$X$ 为从样本空间 $\Omega$ 到实数集 $R$ 的映射,即 $X (\omega):\Omega\rightarrow R$. 如果对任意的 $x\in R$ 均有
		\begin{eqnarray*}
			\{\omega: X (\omega)\le x\}\in \mathcal{F} \mbox{  或  }      \{\omega: X (\omega)< x\}\in \mathcal{F}
		\end{eqnarray*}
		则称 $X (\omega)$ 是可测空间 $(\Omega,\mathcal{F})$ 上的随机变量,简称随机变量.
	\end{defi}
\end{frame}
\begin{frame}
	\frametitle{随机变量两种定义的等价性}
	由定义 \ref{rvdefi2} 推定义 \ref{rvdefi1}: 仅需说明若定义 \ref{rvdefi2} 成立,则对任意 $B\in\mathcal{B}$ 均有
	$$X^{-1}(B):=\{\omega:X(\omega)\in B\}\in \mathcal{F}.$$
	\pause 即只需说明以下集合包含关系成立即可
	\begin{eqnarray*}
		\mathcal{A}:=\{A:X^{-1}(A)\in \mathcal{F}\}\supset \mathcal{B}
	\end{eqnarray*}
	\pause
	欲证上面包含关系成立,我们只需说明以下两点即可:
	\begin{enumerate}[<+-|alert@+>]
		\item $\mathcal{A}$ 是 $\sigma$ 代数;
		\item $O_1:=\{(-\infty,x]:x\in R\}\subset \mathcal{A}$.
	\end{enumerate}
	\pause 再由 $\mathcal{B}:=\sigma (O_1)$ 知 $\mathcal{B}\subset \mathcal{A}$.
\end{frame}
\begin{frame}
	\frametitle{$\mathcal{A}:=\{A:X^{-1}(A)\in \mathcal{F}\}$ 为 $\sigma$ 代数}
	\begin{itemize}[<+-|alert@+>]
		\item $X^{-1}(R)=\{\omega:X (\omega)\in R\}=\Omega\in \mathcal{F}$, 故 $R\in\mathcal{A}$;
		\item 若 $A\in\mathcal{A}$, 即 $X^{-1}(A)\in \mathcal{F}$, 则
		      \begin{eqnarray*}
			      X^{-1}(\overline{A})&=&\pause \{\omega: X(\omega)\in \overline{A}\}\pause =\{\omega:X(\omega)\notin A\}\\
			      &=&\pause \overline{\{\omega:X(\omega)\in A\}}=\pause \overline{X^{-1}(A)}\\ \pause &\in&  \mathcal{F}
		      \end{eqnarray*}
		\item 对于 $A_j\in \mathcal{A}, j=1,2,\cdots,$ 有 $X^{-1}(A_j)\in \mathcal{F},j=1,2,\cdots.$ 从而
		      \begin{eqnarray*}
			      X^{-1}(\cup_{j=1}^\infty A_j)&=&\pause \{\omega:X(\omega)\in\cup_{j=1}^\infty A_j\}
			      =\pause \cup_{j=1}^\infty \{\omega:X(\omega)\in A_j\}\\
			      &=&\pause \cup_{j=1}^\infty X^{-1}(A_j)\\ \pause &\in&\mathcal{F}
		      \end{eqnarray*}

	\end{itemize}
\end{frame}

\begin{frame}{随机变量的分类以及两个注记}
	\begin{itemize}[<+-|alert@+>]
		\item 随机变量 $X$ 是从样本空间 $\Omega$ 到实数 $R$ 的映射,故根据其值域集合可粗略的分为两大类
		      \begin{itemize}[<+-|alert@+>]
			      \item 离散型随机变量:其值域集合是有限点集或可数点集即
			            \[X(\Omega):=\{X(\omega):\omega\in \Omega\}=\{a_n\}_{n\geq 1}\]
			      \item 非离散型随机变量:其值域集合不是有限点集或可数点集
		      \end{itemize}
	\end{itemize}
	\pause
	\begin{rmk} 随机变量的两点注记:
		\begin{itemize}[<+-|alert@+>]
			\item 首先, 随机变量是确定性函数, 自身并没有随机性. 给定样本空间上的样本点, 有唯一确定的实数值与之相对应, 这种对应关系并没有不确定性. 所有的不确定性都体现在样本点是否在试验结果中出现, 和随机变量本身没有关系。随机变量的引入, 更多地是为了数学处理上的方便.
			\item 其次, 随机变量并不是概率论中独有的概念. 若将前面可测映射中的$(\Omega, \mathcal{F}), (E, \mathcal{E})$均取为$(R, \mathcal{B}(\mathbb{R}))$, 则可测映射的定义便退化为实分析中的 ``可测函数". 随机变量是一特殊的可测函数.
		\end{itemize}

	\end{rmk}

\end{frame}












\begin{frame}{示性随机变量}
	\begin{exam}
		设 $\Omega$ 是某随机试验的样本空间,$\mathcal{F}$ 为其事件域 ($\sigma$ 代数),则对于任意的 $A\in \mathcal{F}$, 示性函数
		$I_A(\omega):=\left\{
			\begin{array}{ll}
				0, & \omega\notin A \\
				1, & \omega\in A
			\end{array}\right.$ 是随机变量.
	\end{exam}

	\pause
	\jieda  由示性函数的定义知:
	\begin{eqnarray*}
		\{\omega:I_A(\omega)\le x\}=\left\{
		\begin{array}{ll}
			\pause \emptyset, & x<0,        \\ \pause
			\overline{A},     & x\in [0,1), \\\pause
			\Omega,           & x\ge 1.
		\end{array}
		\right.
	\end{eqnarray*}
	\pause
	显然,无论 $x$ 取何值,均有 $\{\omega:I_A (\omega)\le x\}\in \mathcal{F}$


\end{frame}





\begin{frame}
	\frametitle{随机变量的性质}
	\begin{thm}
		若 $X,Y, \{X_n,n\ge 1\}$ 都为概率空间 $(\Omega,\mathcal{F},P)$ 上的随机变量,则
		\begin{enumerate}[<+-|alert@+>]
			\item $|X|$,$aX+bY,(a,b\in R)$ 均为随机变量;
			\item $X^+:=X\vee 0, X^-:=(-X)\vee 0$ 均为随机变量;
			\item $XY$ 为随机变量;
			\item 若 $X/Y$ 处处有意义,则 $X/Y$ 为随机变量;
			\item $\inf_{n} X_n,\sup_nX_n, \liminf_{n\rightarrow\infty} X_n, \limsup_{n\rightarrow\infty} X_n$ 均为随机变量.
		\end{enumerate}

	\end{thm}

	\pause
	\zheng
	\begin{enumerate}[<+-|alert@+>]
		\item
		      \begin{itemize}[<+-|alert@+>]
			      \item $\{\omega:|X|<x\}=\{\omega:-x<X<x\}=\{\omega:X<x\}\cap\overline{\{\omega:X\le -x\}}\in \mathcal{F}$;
			      \item  $\{\omega:aX<x\}=\{\omega:X<\dfrac{x}{a}\}\in \mathcal{F}$,  (当 $a>0$ 时);
			      \item  设 $Q$ 为有理数集,则
			            \begin{eqnarray*}
				            \{\omega:X+Y<x\}&=&\pause \{\omega:X<x-Y\}=\pause \cup_{r\in Q}\{\omega:X<r<x-Y\}\\
				            &=&\pause \cup_{r\in Q}\{\omega:X<r,Y<x-r\}\\
				            &=&\pause \cup_{r\in Q}(\{\omega:X<r\}\cap \{\omega:Y<x-r\})\\\pause
				            &\in& \mathcal{F}
			            \end{eqnarray*}
			      \item $aX+bY$ 为随机变量显然
		      \end{itemize}
	\end{enumerate}


\end{frame}
\begin{frame}
	\vspace{0.6cm}
	\begin{enumerate}[<+-|alert@+>]
		\setcounter{enumi}{1}
		\item 注意到 $X^+=\dfrac{|X|+X}{2}, X^-=\dfrac{|X|-X}{2}$,易得 $X^+,X^-$ 均为随机变量;
		\item 首先假定 $X,Y$ 非负, 则对任意的 $x>0$ 有
		      \begin{eqnarray*}
			      \{XY<x\}&=&\pause \{X=0\}\cup \{Y=0\}\cup \left(\cup_{r\in Q_+}\left[\{X<r\}\cap \{Y<\frac{x}{r}\}\right]\right)\\ \pause
			      &\in&\mathcal{F}.
		      \end{eqnarray*}
		      \pause
		      对一般的 $X,Y$, 由 $X^+,X^-,Y^+,Y^-$ 为随机变量,可得
		      \begin{eqnarray*}
			      XY=(X^+-X^-)(Y^+-Y^-)=(X^+Y^++X^-Y^-)-(X^+Y^-+X^-Y^+)
		      \end{eqnarray*}
		      为随机变量.
		\item 设 $|Y|>0$ 处处成立,易证 $\dfrac{1}{Y}$ 是随机变量, 故 $\dfrac{X}{Y}=X\cdot \dfrac{1}{Y}$ 为随机变量.
		\item 对任意的 $x\in R$, 我们有
		      \begin{eqnarray*}
			      \{\inf_nX_n<x\}=\cup_n\{X_n<x\}, \quad \{\sup_nX_n\le x\}=\cap_n\{X_n\le x\}
		      \end{eqnarray*}

	\end{enumerate}


\end{frame}

\begin{frame}
	\frametitle{简单随机变量}
	\begin{exam}
		设 $(\Omega,\mathcal{F},P)$ 为一概率空间,$A_i\in\mathcal{F}, i=1,\cdots,n$ 为 $\Omega$ 的一个分割,$a_i,i=1,\cdots, n$ 为 $n$ 个不同的实数,则
		\begin{eqnarray}\label{eq:simplerv}
			X(\omega):=\sum_{i=1}^na_iI_{A_i}(\omega)
		\end{eqnarray}
		作为 $n$ 个示性随机变量的线性组合,仍为随机变量。我们称形如 \eqref{eq:simplerv} 的 $X (\omega)$ 为简单随机变量.
	\end{exam}

\end{frame}

\begin{frame}{随机变量的函数是否仍为随机变量?}
	\vspace{0.5cm}
	\begin{thm}
		设 $X$ 是可测空间 $(\Omega,\mathcal{F})$ 上的随机变量,$g (x)$ 为 $(R,\mathcal{B})\rightarrow (R,\mathcal{B})$ 上的Borel可测函数,证 $Y:=g (X)$ 为 $(\Omega,\mathcal{F})$ 上的随机变量.
	\end{thm}

	\vspace{0.3cm}
	\pause
	\zheng 注意到,对任意的 $B\in \mathcal{B}$, $g^{-1}(B)\in \mathcal{B}$, 故 \pause
	\begin{align*}
		Y^{-1}(B) & =\pause \{\omega:Y(\omega)\in B\}         \\
		          & =\pause \{\omega:g(X(\omega))\in B\}      \\
		          & =\pause \{\omega:X(\omega)\in g^{-1}(B)\} \\
		          & =\pause X^{-1}(g^{-1}(B))                 \\ \pause
		          & \in  \mathcal{F}
	\end{align*}

\end{frame}


\begin{frame}
	\frametitle{随机变量的结构}
	对于 $(\Omega,\mathcal{F},P)$ 上的任意非负随机变量 $X$ 及自然数 $n$,我们可将 $\Omega$ 按 $X$ 的取值进行分割。即令
	\begin{eqnarray*}
		A_k(\omega)&:=&\pause \{\omega:\frac{k}{2^n}\le X(\omega)<\frac{k+1}{2^n}\}, k=0,1,\cdots, n2^n-1\\
		A_{n2^n}(\omega)&:=&\pause \{\omega:X(\omega)\ge n\}
	\end{eqnarray*}
	\pause 则
	\begin{eqnarray*}
		X_n(\omega):=\sum_{k=0}^{n2^n}\frac{k}{2^n}I_{A_k}(\omega)
	\end{eqnarray*}
	为简单随机变量且随机变量序列 $\{X_n,n\ge 1\}$ 满足
	\begin{eqnarray*}
		0\le X_1(\omega)\le X_2(\omega)\le \cdots\le X_n(\omega)\rightarrow X(\omega)
	\end{eqnarray*}
\end{frame}
\begin{frame}
	\frametitle{$X_n (\omega)$ 的单调性}
	% 事实上,要说明 $X_n (\omega)\le X_{n+1}(\omega)$,只需要说明在 $\Omega$ 有限分割集合 $A_k, k=0,1,\cdots,n2^n$ 上均有 $X_n (\omega)\le X_{n+1}(\omega)$.
	注意到对任意的 $k=0,1,\cdots,n2^n-1$,
	\begin{eqnarray*}
		A_k(\omega)&=&\pause \{\omega:\frac{k}{2^n}\le X(\omega)<\frac{k+1}{2^n}\}=\pause \{\omega:\frac{2k}{2^{n+1}}\le X(\omega)<\frac{2(k+1)}{2^{n+1}}\}\\
		&=&\pause \{\omega:\frac{2k}{2^{n+1}}\le X(\omega)<\frac{2k+1}{2^{n+1}}\}\cup \{\omega:\frac{2k+1}{2^{n+1}}\le X(\omega)<\frac{2k+2)}{2^{n+1}}\}\\
		&=&\pause A_k^1(\omega)\cup A_k^2(\omega)
	\end{eqnarray*}
	\pause 故在集合 $A_k (\omega),k=0,1,\cdots,n2^n-1$ 上,\pause
	\begin{eqnarray*}
		X_{n+1}(\omega)=
		\left\{\begin{array}{ll}
			\pause  \frac{2k}{2^{n+1}},   & \omega\in A_k^1(\omega)  \\ \pause
			                              &                          \\
			\pause  \frac{2k+1}{2^{n+1}}, & \omega\in A_k^2(\omega)
		\end{array}
		\right.\pause  (\textcolor{red}{\ge \frac{k}{2^n}=X_n(\omega)})
	\end{eqnarray*}
	\pause
	而在 $A_{n2^n}(\omega):=\{\omega:X (\omega)\ge n\}=\{\omega:X (\omega)\ge \frac{n2^{n+1}}{2^{n+1}}\}$ 上,显然有
	\pause $$X_{n+1}(\omega)\ge n=X_n(\omega).$$
\end{frame}

\begin{frame}
	\frametitle{$X_n (\omega)$ 的收敛性}
	注意到,对任意的 $\omega$, 必定存在 $k$ 使得
	\begin{eqnarray*}
		\frac{k}{2^n}\le X(\omega)<\frac{k+1}{2^n},
	\end{eqnarray*}
	\pause 从而
	\begin{eqnarray*}
		0\le X(\omega)-X_n(\omega)\le \frac{1}{2^n}
	\end{eqnarray*}
	\pause 显然有
	\begin{eqnarray*}
		\lim_{n\rightarrow \infty}X_n(\omega)=X(\omega).
	\end{eqnarray*}
\end{frame}
\begin{frame}
	\frametitle{随机变量的简单随机逼近}
	\begin{thm}
		对 $(\Omega,\mathcal{F},P)$ 上的实值变量 $X (\omega)$ 为随机变量的充要条件是:存在简单随机变量序列 $\{X_n (\omega),n\ge 1\}$ 使得
		\begin{eqnarray*}
			\lim_{n\rightarrow \infty}X_n(\omega)=X(\omega), \quad \forall \omega\in \Omega
		\end{eqnarray*}
		而且当 $X$ 非负时,还可选取 $\{X_n (\omega),n\ge 1\}$ 为非负单调不减的简单随机变量序列.
	\end{thm}
\end{frame}












\begin{frame}{随机变量的生成$\sigma$-代数}
	\begin{defi}
	(\tc{单个随机变量的生成$\sigma$-代数}) 设$X$为定义在$(\Omega,\mathcal{F})$上的随机变量并令
	$\mathcal{F}_X:=\{X^{-1}(A):A\in \mathcal{B}(\mathbb{R})\}$. 我们称$\sigma(X):=\sigma\left(\mathcal{F}_{X}\right)$为随机变量$X$的生成$\sigma$-代数.
	\end{defi}
	\pause

\begin{defi}
  (\tc{有限个随机变量的生成$\sigma$-代数})设$X_k, k=1,2,\cdots, n$为定义在$(\Omega,\mathcal{F})$上的随机变量并令
  $\mathcal{F}_{X_k}:=\{X_k^{-1}(A):A\in \mathcal{B}(\mathbb{R})\}$. 我们称\[\sigma(X_1,X_2,\cdots,X_n):=\sigma\left(\cup_{k=1}^n\mathcal{F}_{X_k}\right)\]为随机变量$X_1,X_2,\cdots,X_n$的生成$\sigma$-代数.
\end{defi}
\pause
\begin{itemize}[<+-|alert@+>]
	\item $\mathcal{F}_{X}, \mathcal{F}_{X_k}$均为$\sigma$-代数, 但$\cup_{k=1}^n\mathcal{F}_{X_k}$不一定是$\sigma$-代数, 因此
	\[\sigma(X)=\mathcal{F}_X, \quad \mbox{ 但  } \quad  \sigma(X_1,X_2,\cdots,X_n)\neq \cup_{k=1}^n\mathcal{F}_{X_k}.\]
	\item 若$X$是$(\Omega,\mathcal{F})$上的随机变量,则必有$\sigma(X):=\mathcal{F}_X\subset\mathcal{F}$, 即$\sigma(X)$是$\Omega$上使得$X$成为随变量所需要的最小$\sigma$-代数.
	\item 类似的, $\sigma(X_1,X_2,\cdots,X_n)$是$\Omega$上使得$X_1,X_2,\cdots,X_n$为随机变量所需要的最小$\sigma$-代数.
	\item 随机变量$X$生成的$\sigma-$代数$\sigma(X)$集中体现了$X$的取值信息.
\end{itemize}
\end{frame}
\begin{frame}{两个随机变量之间的关系}
\begin{defi}
考虑两个随机变量$X, Y$,如果对任意的$B\in\mathcal{B}(\mathbb{R})$均有
\[
Y^{-1}(B) \in \sigma(X),\ \mbox{等价的}\ \sigma(Y)\subset \sigma(X).
\]则称$Y$适应(adaptive to) $X$.
\end{defi}

\pause

\begin{thm}
考虑可测空间$(\Omega, \mathcal{F})$, $X$和$Y$为定义其上的随机变量. 若$Y$适应$X$, 则存在可测函数$g: \mathbb{R} \rightarrow \mathbb{R}$, 使得
\[
Y(\omega)=g(X(\omega)),  \ \forall \omega \in \Omega .
\]
\end{thm}

\pause

\begin{rmk}
	\begin{itemize}[<+-|alert@+>]
		\item 随机变量的生成$\sigma$-代数研究随机变量间关系起着重要作用.
		\item 直观地看,$Y$适应$X$, 意味着$Y$包含的信息被$X$包含的信息所涵盖. 换句话说,$Y$和$X$间存在导出关系.
	\end{itemize}
\end{rmk}



\end{frame}
\subsection{随机向量}

\begin{frame}{随机向量: 如何定义$\mathcal{B}(\mathbb{R}^n)$}
\begin{itemize}
	\item 直观上来讲, 随机向量就是取值于$\mathbb{R}^n$的随机变量.
	\item 如何定义$\mathcal{B}(\mathbb{R}^n)$
	\begin{itemize}[<+-|alert@+>]
		\item 注意到$$\mathbb{R}^n:=\mathbb{R}\times \mathbb{R}\times\cdots\times \mathbb{R}=\Pi_{k=1}^n\mathbb{R}$$
		\item 我们希望
		\[\mathcal{B}(\mathbb{R}^n)=\mathcal{B}(\mathbb{R})\times \mathcal{B}(\mathbb{R})\times\cdots\times\mathcal{B}(\mathbb{R})=\Pi_{k=1}^n\mathcal{B}(\mathbb{R})\]
		\item 但$\Pi_{k=1}^n\mathcal{B}(\mathbb{R})$不是$\sigma$-代数.
		\item 因此, 定义
		\[\mathcal{B}(\mathbb{R}^n):=\sigma(\mathcal{B}(\mathbb{R})\times \mathcal{B}(\mathbb{R})\times\cdots\times\mathcal{B}(\mathbb{R}))=\sigma(\Pi_{k=1}^n\mathcal{B}(\mathbb{R}))\]
		\item 我们也可以类比$\mathcal{B}(\mathbb{R})$的定义方法, 先定义$\mathbb{R}^n$上的立方体
		$$I=I_1\times I_2\times\cdots\times I_n=\Pi_{k=1}^nI_k, \ I_k=(a_k, b_k].$$ 然后定义$\mathcal{B}(\mathbb{R}^n)$为:
		\[\mathcal{B}(\mathbb{R}^n):=\sigma(\{I:I\mbox{为}\mathbb{R}^n\mbox{上的立方体}\}).\]%集合的生成$\sigma$-代数来定义$\mathcal{B}(\mathbb{R}^n)$.
	\end{itemize}

\end{itemize}

\end{frame}

\begin{frame}{随机向量}
\begin{defi}
(\tc{$n$维随机向量}) 考虑可测空间$(\Omega,\mathcal{F})$, 若映射$X: \Omega \rightarrow \mathbb{R}^{n}$使得对任意的$B \in \mathcal{B}(\mathbb{R}^{n})$均有
	\[
	X^{-1}(B) \in \mathcal{F}
	\]
则称$X$为$n$维随机向量.
\end{defi}
\pause
\begin{rmk}
若记$X=(X_1, X_2,\cdots, X_n)$, 则$X$为$n$维随机向量当且仅当$X_k, k=1,\cdots,n$为随机变量.
\end{rmk}

\pause
\begin{defi}
(\tc{复随机变量}) 考虑可测空间$(\Omega,\mathcal{F})$以及映射$Z$
\[
Z(\omega)=X(\omega)+\mathrm{i} Y(\omega) \in \mathbb{C}
\]
其中$\mathrm{i}=\sqrt{-1}$是虚数单位. 如果$(X(\omega), Y(\omega))$构成二维随机向量, 则称$Z$为复随机变量.
\end{defi}

\end{frame}



