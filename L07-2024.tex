

\title[概率论]{第七讲: 事件的独立性}
%\author[张鑫 {\rm Email: xzhangseu@seu.edu.cn} ]{\large 张 鑫}
\institute[东南大学数学学院]{\large \textrm{Email: xzhangseu@seu.edu.cn} \\ \quad  \\
	\large 东南大学\quad 数学学院\\
	\vspace{0.3cm}
	%\trc{ 公共邮箱: \textrm{zy.prob@qq.com}\\
		% \hspace{-1.7cm}  密\qquad 码: \textrm{seu!prob}}
}
\date{}

{\setbeamertemplate{footline}{}
	\begin{frame}
		\titlepage
	\end{frame}
}
\subsection{事件的独立性}

\begin{frame}
  \frametitle{两个事件的独立性}
  \begin{itemize}[<+-|alert@+>]
  \item 直观上来说,两个事件的独立性是指:一个事件的发生不影响另一个事件的发生。比如在掷两颗骰子的实验中,第一颗骰子的点数和第二颗骰子的点数是互不影响的.
  \item 从概率的角度看,$P (A|B)$ 与 $P (A)$ 的差别在于:事件 $B$ 的发生改变了事件 $A$ 发生的概率,也即事件 $B$ 对事件 $A$ 有某种影响。故如果 $A$ 与 $B$ 的发生是相互不影响的,则有 \pause
    \begin{eqnarray*}
      P(A|B)=P(A),\quad P(B|A)=P(B).
    \end{eqnarray*}
    \pause 上面两式均等价于
    \begin{eqnarray}\label{eq:inden}
      P(AB)=P(A)P(B)
    \end{eqnarray}
  \item 注意到 \eqref{eq:inden} 式对 $P (B)=0$ 或 $P (A)=0$ 仍然成立,为此,我们用 \eqref{eq:inden} 作为两个事件相互独立的定义.
  \end{itemize}
\end{frame}

\begin{frame}
  \frametitle{两个事件独立性的定义}
  \begin{defi}
    如果对事件 $A$ 与 $B$ 有
    \begin{eqnarray*}
      P(AB)=P(A)P(B)
    \end{eqnarray*}
    成立,则称 \textcolor{red}{事件 $A$ 与 $B$ 相互独立}, 简称 \textcolor{red}{$A$ 与 $B$ 独立}. 否则称 $A$ 与 $B$ 不独立或相依.
  \end{defi}

\begin{rmk}
	\begin{itemize}
		\item 零概率事件 $E$ 与任何事件相互独立,特别的,不可能事件与任何事件相互独立
        \item 非零概率互不相容事件,一定不独立
        \item 非零概率相互独立事件,一定相容
\end{itemize}


\end{rmk}

\pause
  \textcolor{red}{如何确定事件的独立性: }
  \begin{itemize}[<+-|alert@+>]
  \item 实际问题中,两个事件的独立大多根据经验及相互有无影响的直观性来判断.
  \item 但对于较复杂事件,有无相互影响并不是很直观,则需要验证 \eqref{eq:inden} 式是否成立来说明独立性.
  \end{itemize}
\end{frame}


 \begin{frame}
	\frametitle{对立事件的独立性}
	\begin{thm}
		若 $A$ 与 $B$ 独立,则 $A$ 与 $\overline{B}$ 独立,$\overline{A}$ 与 $B$ 独立,$\overline{A}$ 与 $\overline{B}$ 独立.
	\end{thm}
	\pause

	\zheng 我们仅证 $P (A\overline{B})=P (A) P (\overline{B})$, 其余类似可证.
	\begin{eqnarray*}
		P(A\overline{B})&=&\pause P(A-B)=\pause P(A)-P(AB)\pause =P(A)-P(A)P(B)\\
		&=&\pause P(A)(1-P(B))=\pause P(A)P(\overline{B})
	\end{eqnarray*}

	\pause
	对于上面的定理直观上来理解也是很容易的:因 $A,B$ 独立,故 $A$ 的发生不影响 $B$ 的发生,从而也不会影响 $B$ 的不发生,$\cdots$
\end{frame}









   \begin{frame}
	\frametitle{三个事件的独立性}
	\begin{defi}
		设 $A,B,C$ 三个事件,如果有
		\begin{eqnarray}\label{eq:inden1}
			\left.\begin{array}{l}
				P(AB)=P(A)P(B)\\
				P(AC)=P(A)P(C) \\
				P(BC)=P(B)P(C)
			\end{array}\right\}\\
			\label{eq:inden2}
			P(ABC)=P(A)P(B)P(C)
		\end{eqnarray}
		则称 $A,B,C$ 相互独立。如果仅有 \eqref{eq:inden1} 式成立,则称 $A,B,C$ 两两独立.
	\end{defi}
\end{frame}

\begin{frame}
	\frametitle{两两独立与相互独立的关系}
	\begin{itemize}[<+-|alert@+>]
		\item 由定义可知,三个事件相互独立必能推出两两独立.
		\item 但两两独立未必能推出相互独立,即 \eqref{eq:inden1} 式成立,不一定能推出 \eqref{eq:inden2} 成立
		\begin{itemize}
			\item 考虑独立投掷两枚均匀硬币的随机试验,设事件 $A$ 代表第一枚硬币正面朝上,事件 $B$ 代表第二枚硬币正面朝上,事件 $C$ 表示两枚硬币结果相同。易知: \pause
			$A$ $B$ 和 $C$ 是两两独立,但
			\begin{align*}
				P(A\cap B\cap C)=1/4\neq 1/8=P(A)P(B)P(C).
			\end{align*}
			\item 考虑一个均匀的正四面体,第一二三面分别染上红 / 白 / 黑色,第四面同时染上红白黑色。现在以 $A,B,C$ 分别记投一次四面体出现红,白,黑色朝下的事件。则易有 \pause
			\begin{eqnarray*}
				P(A)=P(B)=P(C)=\pause 1/2\\ \pause
				P(AB)=P(BC)=P(AC)=\pause 1/4\\ \pause
				P(ABC)=\pause 1/4    \pause
			\end{eqnarray*}
		\end{itemize}\vspace{-0.7cm}
	\end{itemize}
\end{frame}

		\begin{frame}
			\frametitle{两两独立与相互独立的关系}
			\begin{itemize}[<+-|alert@+>]
		\item 反之,如果 \eqref{eq:inden2} 成立,是否能推出 \eqref{eq:inden1} 成立?
		\begin{itemize}[<+-|alert@+>]
			\item 考虑一个均匀的正八面体,第 1, 2, 3, 4 面染上红色,第 1, 2, 3, 5 面染上白色,第 1, 6, 7, 8 面染上黑色。现在以 $A,B,C$ 分别记投一次八面体出现红,白,黑色朝下的事件,则 \pause
			\begin{eqnarray*}
				P(A)=P(B)=P(C)=\pause 4/8=1/2\\ \pause
				P(ABC)=\pause 1/8   \\ \pause
				P(AB)=3/8\pause \neq 1/4=P(A)P(B)
			\end{eqnarray*}
		\end{itemize}
	\end{itemize}
\end{frame}

\begin{frame}
	\frametitle{三个以上事件的独立性}
	\begin{defi}
		设 $(\Omega,\mathcal{F}, P)$ 为一概率空间,$A_1,A_2,\cdots,A_n\in\mathcal{F}$, 对任意的 $1\le k\le n$ 及任意的 $1< j_1<j_2<\cdots<j_k\leq n$ 均有:
		\begin{eqnarray}\label{eq:mulinden0}
			P(A_{j_1}A_{j_2}\cdots A_{j_k})=P(A_{j_1})P(A_{j_2})\cdots P(A_{j_k})
		\end{eqnarray}
		成立,则称事件 $A_1,\cdots, A_n$ 相互独立.
	\end{defi}
	\pause
	\begin{itemize}[<+-|alert@+>]
		\item \eqref{eq:mulinden0} 式共有多少个等式?\pause
		\begin{eqnarray}
			\label{eq:mulinden}
			\left.\begin{array}{l}
				P(A_{j_1}A_{j_2})=P(A_{j_1})P(A_{j_2})\\
				P(A_{j_1}A_{j_2}A_{j_3})=P(A_{j_1})P(A_{j_2})P(A_{j_3}) \\
				\qquad \vdots\\
				P(A_1A_2\cdots A_n)=P(A_1)P(A_2)\cdots P(A_n)
			\end{array}\right\} \pause \textcolor{red}{C_n^2+\cdots+C_n^n=2^n-n-1}
		\end{eqnarray}
		\pause
		\item 从定义可以看出,$n$ 个相互独立事件中的任取 $m$($2\le m\le n$) 个事件仍是相互独立的,而且任意一部分与另一部分也是独立的.
		\item 类似于前面的证明,将相互独立事件中的任一部分换为对立事件,所得诸事件仍是相互独立的.
	\end{itemize}



\end{frame}

\begin{frame}
	\frametitle{任意多个事件相互独立}
	\begin{defi}
		设 $(\Omega,\mathcal{F}, P)$ 为一概率空间,每个 $t\in T$ 有 $A_t\in \mathcal{F}$. 称 $\{A_t, t\in T\}$ 为独立事件族,如果对 $T$ 的任意有限子集 $\{t_1,t_2,\cdots, t_s\}$, 事件 $A_{t_1}, A_{t_2},$ $\cdots, A_{t_s}$ 相互独立.
	\end{defi}


	\vspace{0.8cm}
	\pause
	\begin{exam}
		$\mathcal{F}$ 中事件序列 $\{A_n\}$ 为相互独立的充分必要条件是,任意 $n\geq 1$, 事件 $A_1,A_2, \cdots, A_n$ 独立;等价的,任意有限个自然数 $k_1,\cdots, k_s$ 有
		\begin{eqnarray*}
			P(A_{k_1}A_{k_2}\cdots A_{k_s})=P(A_{k_1})P(A_{k_2})\cdots P(A_{k_s})
		\end{eqnarray*}

	\end{exam}

\end{frame}


\begin{frame}{条件独立}
\begin{defi}
称事件 $A$ 和 $B$ 是关于 $E$ 条件独立的,如果 $$P (A\cap B|E)=P (A|E) P (B|E)$$
\end{defi}
\pause
\begin{itemize}[<+-|alert@+>]
\item 两个事件可以在给定事件 $E$ 的条件下是条件独立的,但它们不是独立的.
\item 两个事件可以是独立但却不是关于 $E$ 条件独立的.
\item 两个事件可以关于 $E$ 条件独立但关于 $E^c$ 不存在条件独立.
\end{itemize}
\end{frame}

\begin{frame}{条件独立不意味着独立}
\begin{exam}
假设有两枚硬币,一枚是均匀的,一枚是不均匀的。从两枚硬币中随机的选一枚硬币并进行抛掷 2 次,若令
\begin{align*}
  F &:=\{\mbox{选取的硬币是均匀的}\}\\
   A_{1}&:= \{\mbox{第一次投掷硬币正面朝上}\}\\
   A_{2}&:= \{\mbox{第二次投掷硬币正面朝上}\}
\end{align*}
则给定 $F$ 为条件,$A_1$ 和 $A_2$, 是相互独立的,$A_1$ 和 $A_2$ 并不是无条件独立的,因为 $A_1$ 会提供关于 $A_2$ 的信息.
\end{exam}
\end{frame}

\begin{frame}{独立不意味着条件独立}
\begin{exam}
假设只有我的朋友 Alice 和 Bob 给我打过电话。每天他俩都会相互独立地决定是否给我打电话。若令
\begin{align*}
	 A&:= \{\mbox{Alice 给我打电话}\}\\
	 B&:= \{\mbox{Bob 给我打电话}\}\\
     R&:=\{\mbox{听到电话铃响}\}
  \end{align*}
  \pause
  \begin{itemize}[<+-|alert@+>]
  \item 显然,$A$ 和 $B$ 是无条件独立的.
  \item 但现在我听到一声电话铃响,那 $A$ 和 $B$ 就不再独立了:如果这个电话不是 Alice 打的,那就肯定是 Bob 打的。从而 \pause
  $$P(B|R)<1=P(B|A^cR)= \frac{P(BA^cR)}{P(A^cR)}= \frac{P(BA^c|R)}{P(A^c|R)}.$$
  显然: $P (BA^c|R)>P (B|R) P (A^c|R)$
  \item $B$ 与 $A^c$ 关于 $R$ 不条件独立,$A,B$ 亦是如此.
  \end{itemize}

\end{exam}
\end{frame}

\begin{frame}{给定 $E$ 条件独立 $vs$ 给定 $E^c$ 条件独立}
\begin{exam}
\label{27}
假设有两种课程:好的课程和坏的课程。在好的课上,如果你努力,就很有可能得到 $A$. 在坏的课上,教授随机分配给学生分数,而不管他们是否努力。若令
\begin{align*}
	G&:= \{\mbox{这个课程是好的}\}\\
	W&:= \{\mbox{你学习努力}\}\\
	A&:=\{\mbox{你的得分为} A\}
 \end{align*}
 \pause 这时,给定 $G^c$, $A$ 和 $W$ 是条件独立的,但给定 $G$, $A$ 和 $W$ 却不是独立的!

\end{exam}
\end{frame}

\subsection{随机试验的独立性}
 \begin{frame}
              \frametitle{随机试验的独立性}

               \begin{itemize}[<+-|alert@+>]
               \item 先考虑两个随机试验,假定 $(\Omega_i,\mathcal{F}_i,P_i), i=1,2$ 为第 $i$ 个随机试验对应的概率空间。按照之前独立性的理解,两个试验的独立性应当叙述为:\pause
                \textcolor{red}{ \begin{eqnarray*}
                   &&\mbox{对任何的} A_i\in\mathcal{F}_i, i=1,2, A_1\mbox{与} A_2\mbox{同时}\\
                   &&\mbox{发生的概率等于它们各自概率之乘积}
                 \end{eqnarray*}}
             \item 两个不妥:
               \begin{itemize}[<+-|alert@+>]
               \item ``$A_1$ 与 $A_2$ 同时发生" 应当是这两个事件的交,但它们分别是两个样本空间 $\Omega_1,\Omega_2$ 的子集,无法进行运算;
               \item 两个概率空间有各自的概率 $P_1, P_2$, 但涉及两个度验,命题中 ``同时发生的概率" 既不能用 $P_1$ 也不能用 $P_2$ 来度量.
               \end{itemize}
             \item 解决方法:构造可以同时描述两个试验的新概率空间 $(\Omega,\mathcal{F},P)$.
               \end{itemize}
             \end{frame}

             \begin{frame}
               \frametitle{乘积空间的构造}
               \begin{itemize}[<+-|alert@+>]
               \item 样本乘积空间: $\Omega:=\Omega_1\times \Omega_2=\{(\omega_1,\omega_2):\omega_1\in\Omega_1\mbox{且}\omega_2\in \Omega_2\}$;
               \item 乘积 $\sigma$- 代数 $\mathcal{F}_1\times\mathcal{F}_2$:
                 \begin{itemize}[<+-|alert@+>]
                 \item 可测矩形集类: $\mathcal{G}:=\{A_1\times A_2: A_1\in\mathcal{F}_1, A_2\in \mathcal{F}_2\}$;
                 \item $\mathcal{F}_1\times \mathcal{F}_2:=\sigma(\mathcal{G})$
                 \end{itemize}
               \item 乘积概率测度:
                 \begin{itemize}[<+-|alert@+>]
                 \item 对于每个可测矩形 $A_1\times A_2\in \mathcal{G}$ 定义如下集函数:
                   \begin{eqnarray}\label{eq:timeprob}
                     P(A_1\times A_2)=P_1(A_1)P_2(A_2), \quad A_i\in\mathcal{F}_i, i=1,2.
                   \end{eqnarray}
                 \item 理论上可以证明如上定义在 $\mathcal{G}$ 上的集函数 $P$ 可唯一地扩张为 $\mathcal{F}_1\times\mathcal{F}_2$ 上的概率测度,称之为 $P_1$ 与 $P_2$ 的乘积 (概率) 测度.
                 \end{itemize}
               \item 在上述乘积测度下
                 \begin{eqnarray*}
                   &&P(A_1\times \Omega_2)=P_1(A_1), \quad P(\Omega_1\times A_2)=P_2(A_2)\\\pause
                   &&\pause P\big((A_1\times \Omega_2)\cap (\Omega_1\times A_2)\big)=\pause P(A_1\times A_2)\\
                   &&\pause = P_1(A_1)P_2(A_2)=\pause P(A_1\times\Omega_1)P(\Omega_1\times A_2)
                 \end{eqnarray*}
               \item $(\Omega_i,\mathcal{F}_i,P_i)$ 的独立性取决于乘积样本空间 $\Omega_1\times\Omega_2$ 上的概率是否取作由 (\ref{eq:timeprob}) 所确定的乘积测度
               \end{itemize}
             \end{frame}

             \begin{frame}
               \frametitle{$n$ 个试验相互独立的定义}

               \begin{defi}
                 设有 $n$ 个随机试验,第 $i$ 个试验的概率空间为 $(\Omega_i,\mathcal{F}_i,P_i),$ $ i=1,\cdots,n$. 代表这 $n$ 个试验的乘积样本空间 $\Omega=\Omega_1\times \cdots \times \Omega_n$, $\mathcal{F}=\mathcal{F}_1\times \cdots\times \mathcal{F}_n=\sigma (\mathcal{G})$, 其中 $\mathcal{G}$ 为形如 $B_1\times\cdots\times B_n (B_i\in\mathcal{F}_i)$ 的可测矩形的全体。如果 $(\Omega,\mathcal{F})$ 上的概率测度 $P$ 是 $P_1,\cdots, P_n$ 的乘积测度,即对任何 $B_1\times\cdots\times B_n\in \mathcal{G}$ 满足
                 \begin{eqnarray*}
                   P(B_1\times\cdots\times B_n)=P_1(B_1)\cdots P(B_n),
                 \end{eqnarray*}
                 则称这 $n$ 个度验独立. \pause 如果现设 $$\Omega_i\equiv \Omega_0, \mathcal{F}_i\equiv \mathcal{F}_0, P_i\equiv P_0, i=1,\cdots,n, $$ 即 $n$ 个试验有相同的概率空间,则称它们为 $n$ 个 (重) 独立重复试验. \pause 如果在 $n$ 个独立重复实验中,每次试验的可能结果为两个:$A$ 或 $\overline{A}$, 则称这种试验为 \textcolor{red}{$n$ 重伯努利试验}.
               \end{defi}
             \end{frame}
                          \begin{frame}{彩票问题}
               % \frametitle{试验的独立性}
               % \begin{defi}
               %   设有两个实验 $E_1$ 和 $E_2$,假如实验 $E_1$ 的任一结果(事件)与试验 $E_2$ 的任一结果(事件)都是相互独立事件,则称这两个实验相互独立.
               % \end{defi}
               % \pause
               % \begin{defi}
               %   如果 $n$ 个实验 $E_1,E_2,\cdots,E_n$ 的任一结果都是相互独立的事件,则称试验 $E_1,\cdots E_n$ 相互独立。如果这 $n$ 个实验是相同的,则称其为 \textcolor{red}{$n$ 重独立重复实验}. 如果在 $n$ 重独立重复实验中,每次试验的可能结果为两个:$A$ 或 $\overline{A}$, 则称这种试验为 \textcolor{red}{$n$ 重伯努利试验}.
               % \end{defi}
               % \pause

               \begin{exam}
                 某彩票每周开奖一次,每次提供十万分之一的中奖机会,且各周开奖是独立的。若你每周买一张彩票,坚持十年(每年按 52 周计算),试求未中奖的概率.
               \end{exam}

               \pause  \jieda 依假设,每次中奖的概率为 $10^{-5}$, 于是每次不中奖的概率是 $1-10^{-5}$. 另外十年一共购买 520 次彩票,而每次开奖都是独立的,相当于进行了 520 次独立重复试验. \pause 若记 $A_i$ 为 “第 $i$ 次开奖不中奖”, 则 $A_1,\cdots, A_{520}$ 相互独立,从而
               \begin{eqnarray*}
                 P(A_1A_2\cdots A_{520})=(1-10^{-5})^{520}=0.9948
               \end{eqnarray*}

             \end{frame}













\subsection{全概率公式的应用}
\begin{frame}{全概率公式的应用}
	\begin{exam}
		有三个罐子, 各装有两个球, 分别为两个白球、一白一黑和两个黑球. 任意取出一个罐子, 摸出一球, 发现是白球. (1)求该罐中另一个球也是白球的概率; (2)把摸出的球放回罐中, 再从该罐中随机摸出一球, 求该球也是白球的概率.
	\end{exam}

	\begin{jieda}
		\begin{itemize}[<+-|alert@+>]
			\item $A_k, k=1,2$:表示第$k$次取球取出的是白球的事件;
			\item $B_k, k=1,2,3$:表示取出的是装有两白、一白一黑和两黑球的罐子;
			\item 问题(1)要求的是该白球取自两白的罐子的概率, 即$P(B_1|A_1)$;
			\item 由条件概率公式, 得$P(B_1|A_1)=\frac{P(A_1B_1)}{P(A_1)}=\frac{P(B_1)P(A_1|B_1)}{P(A_1)};$%=\frac{1/3}{1/2}=\frac{2}{3}.
			\begin{itemize}[<+-|alert@+>]
				\item $P(B_1)=1/3$, \pause $P(A_1|B_1)=1$;\pause
				\item $P(A_1)=\sum\limits_{k=1}^{3}P(B_k)P(A_1|B_k)=\dfrac{1}{3}\cdot 1+\dfrac{1}{3}\cdot\dfrac{1}{2}+\dfrac{1}{3}\cdot 0=\dfrac{1}{2}.$
			\end{itemize}
			\item $P(B_1|A_1)=\dfrac{1/3\cdot 1}{1/2}=2/3$.
		\end{itemize}

	\end{jieda}
\end{frame}

\begin{frame}{全概率公式的应用}
	\begin{itemize}[<+-|alert@+>]
	\item 问题(2)是在同一个罐子两次有放回的取球, 要求的是在第一次取出白球的条件下, 第二次取出的还是白球的条件概率, 即$P(A_2|A_1)$
	\item 由条件概率的定义, 知\pause $P(A_2|A_1)=\frac{P(A_1A_2)}{P(A_1)}$;\pause %$=\frac{5/12}{1/2}=\frac{5}{6}.$为此, 先要用全概率公式求出$P(A_1A_2)$:
	\begin{itemize}[<+-|alert@+>]
		\item $P(A_1)=\sum\limits_{k=1}^{3}P(B_k)P(A_1|B_k)=\dfrac{1}{3}·1+\dfrac{1}{3}·\dfrac{1}{2}+\dfrac{1}{3}·0=\dfrac{1}{2}.$
		\item $P(A_1A_2)=\sum\limits_{k=1}^{3}P(B_k)P(A_1A_2|B_k)=\dfrac{1}{3}·1+\dfrac{1}{3}·\dfrac{1}{4}+\dfrac{1}{3}·0=\dfrac{5}{12}.$
	\end{itemize}
	\item 	$P(A_2|A_1)=\frac{P(A_1A_2)}{P(A_1)}=\dfrac{5/12}{1/2}=\dfrac{5}{6}.$
\end{itemize}
\end{frame}
\begin{frame}{摸球问题}
	\begin{exam}
		甲盒中有球$5$红$1$黑, 乙盒中有球$5$红$3$黑. 随机取出一个盒子, 从中无放回地相继取出两个球, 试求在第一个球是红球的条件下, 第二个球也是红球的概率.
	\end{exam}
	\pause

	\begin{jieda}
		\begin{itemize}[<+-|alert@+>]
			\item $B:=$\{第一个球是红球\}, $C:=$\{第二个球是红球\}
			\item $P(C|B)=\pause \dfrac{P(BC)}{P(B)}$ \pause
			\begin{itemize}[<+-|alert@+>]
				\item $A:=$\{取出的是甲盒\}
				\item $P(BC)=\pause P(A)P(BC|A)+P(\overline{A})P(BC|\overline{A})=\pause \dfrac{1}{2}·\dfrac{5}{6}·\dfrac{4}{5}+\dfrac{1}{2}·\dfrac{5}{8}·\dfrac{4}{7}=\dfrac{43}{84}$


				\item $P(B)=\pause P(A)P(B|A)+P(\overline{A})P(B|\overline{A})=\pause \dfrac{1}{2}·\dfrac{5}{6}+\dfrac{1}{2}·\dfrac{5}{8}=\dfrac{35}{48}$%于是$\overline{A}$即为取出的是乙盒的事件. 由条件概率公式和全概率公式知
				%			\begin{align*}
					%				P(C|B)&=\frac{P(BC)}{P(B)}=\frac{P(A)P(BC|A)+P(\overline{A})P(BC|\overline{A})}{P(A)P(B|A)+P(\overline{A})P(B|\overline{A})}\\
					%				&=\frac{\dfrac{1}{2}·\dfrac{5}{6}·\dfrac{4}{5}+\dfrac{1}{2}·\dfrac{5}{8}·\dfrac{4}{7}}{\dfrac{1}{2}·\dfrac{5}{6}+\dfrac{1}{2}·\dfrac{5}{8}}=\frac{172}{245}.
					%			\end{align*}
			\end{itemize}
			\item $P(C|B)=\dfrac{43}{84}/\dfrac{35}{48}=\dfrac{172}{245}$
		\end{itemize}

	\end{jieda}
	%	\begin{itemize}
		%		\item 在该例的计算中, 分子与分母都用到了全概率公式
		%	\end{itemize}
\end{frame}


%\begin{frame}{全概率公式的应用}
%	\begin{jieda}
%		于是由全概率公式知
%		$$P(E)=P(B)P(E|B)+P(B^c)P(E|B^c)=\frac{1}{2}\left(P(A_0)+P(A_1)+P(A_2)\right)=\frac{1}{2}.$$
%	\end{jieda}
%	\begin{itemize}
%		\item 应当注意, 根据对称性, 我们可以得出: 乙抛出的反面比甲多的概率也是$\dfrac{1}{2}$; 而不是: 乙抛出的正面比甲少的概率等于$\dfrac{1}{2}$.
%	\end{itemize}
%\end{frame}
\begin{frame}
	\frametitle{一般摸球模型}
	\begin{exam}
		袋中有$r$个红球与$b$个黑球. 每次从袋中任摸出 1 球并连同$s$个同色球一起放回. 以$R_n$表示第$n$次摸出红球, 试证$P(R_n)=\dfrac{r}{r+b}$.
	\end{exam}

	\pause
	\zheng 利用归纳法来证明:$n=1$时, $P(R_1)=\dfrac{r}{r+b}$显然成立.

	\pause 假设$n-1$时命题成立. 为求$P(R_n)$, 我们以第 1 次取球的可能结果$R_1$与$\bar{R}_1$作为$\Omega$的分割, 用全概率公式可得:
	\pause
	\begin{align*}
		P(R_n)&=\pause P(R_1)P(R_n|R_1)+P(\bar{R}_1)P(R_n|\bar{R}_1)\\
		&= \pause\dfrac{r}{r+b}\cdot \dfrac{r+s}{r+s+b}+\dfrac{b}{r+b}\cdot \dfrac{r}{r+b+s}\\
		&=\pause\dfrac{r}{r+b}
	\end{align*}
	\pause
	\begin{rmk}
		当$s=0$时相当于放回摸球, 而$s=-1$相当于不放回摸球.
	\end{rmk}

\end{frame}




\begin{frame}{Simpson 悖论}
	\begin{exam}
		有两种治疗肾结石的方案,其治疗效果如下:
		\begin{itemize}[<+-|alert@+>]
			\item \textcolor{cyan}{方案$1$:} 小结石患者占$25\%$, 大结石患者占$75\%$, 小结石患者的治愈率是$93\%$, 大结石患者的治愈率是$73\%$
			\item \textcolor{cyan}{方案$2$:} 小结石患者占$77\%$, 大结石患者占$23\%$, 小结石患者的治愈率是$87\%$, 大结石患者的治愈率是$69\%$
			\item 不管是对小结石患者, 还是大结石患者, 方案$1$的治愈率都要高于方案$2$
			\item \textcolor{red}{方案$1$优于方案$2$吗?}
		\end{itemize}
	\end{exam}
	\pause
	\begin{jieda}
		计算两种方案的治愈率
		\begin{itemize}[<+-|alert@+>]
			\item 记$A$=\{患者是小结石患者\}, $B$=\{患者被治愈\}
			\item 根据全概率公式\pause
			\begin{align*}
				&\hspace{-0.5cm}P_1(B)=\pause P_1(A)P_1(B|A)+P_1(\overline{A})P_1(B|\overline{A})=\pause 0.25·0.93+0.75·0.73=0.78\pause \\
				&\hspace{-0.5cm} P_2(B)=\pause P_2(A)P_2(B|A)+P_2(\overline{A})P_2(B|\overline{A})=\pause 0.77·0.87+0.23·0.69=0.8286
			\end{align*}
			\item 方案$2$的治愈率高于方案$1$, 可见方案$1$并不优于方案$2$
		\end{itemize}
	\end{jieda}
\end{frame}

\begin{frame}
	\frametitle{敏感性问题调查}
	敏感性问题调查方案的关键在于要使被调查者愿意作出真实的回答又能保守个人秘密, 如果调查方案有误, 被调查者就会拒绝配合, 所得调查数据将失去真实性.
	\pause
	\vspace{0.5cm}

	{\bf 调查方案: } 被调查者只需要加答以下两个问题中的一个问题,且只需要回答“是”或“否”.

	\pause
	问题 A: 你的生日是否在 7 月 1 日之前?

	问题 B: 所调查的敏感性问题.

	\pause
	\vspace{0.5cm}
	{\bf 调查方案的操作: }

	( 1)被调查者在没有旁人的情况下独自一人在房间内操作回答问题

	\pause
	( 2)被调查者从一个罐子中随机抽一球, 看过颜色后放回, 若抽到白球, 回答问题 A, 抽到红球, 回答问题 B.

	( 3)被调查者无论回答问题 A 还是问题 B, 只需在仅有“是”与“否”选项的答卷上作答然后将答卷放入密封的投票箱内.

\end{frame}

\begin{frame}
	\frametitle{敏感性问题调查}
	{\bf \textcolor{red}{问题:}} 假若我们有$n$张问卷, 其中$k$张回答“是”, 我们如何确定选定红球回答问题 B 为“是”的概率?


	\pause
	{\bf \textcolor{red}{已知:}}
	\begin{itemize}[<+-|alert@+>]
		\item 红白球的比例, 即$P(\mbox{红球}):=\pi, P(\mbox{白球})=1-\pi$;
		\item $P(\mbox{是}|\mbox{白球})=0.5$;
		\item $P(\mbox{是})\approx \dfrac{k}{n}$.
	\end{itemize}
	\pause
	{\bf \textcolor{red}{待求:}} $p:=P(\mbox{是}|\mbox{红球})$.

	\pause
	{\bf\jieda} 由全概率公式可得
	\begin{eqnarray*}
		P(\mbox{是})=P(\mbox{白球})P(\mbox{是}|\mbox{白球})+P(\mbox{红球})P(\mbox{是}|\mbox{红球})
	\end{eqnarray*}
	即\pause
	\begin{eqnarray*}
		\dfrac{k}{n}\approx (1-\pi)\cdot 0.5+\pi\cdot p \Rightarrow p=\dfrac{k/n-0.5(1-\pi)}{\pi}
	\end{eqnarray*}


\end{frame}



\begin{frame}{掷硬币问题}
	\begin{exam}
		甲、乙二人抛掷一枚均匀的硬币, 甲抛了$100$次, 乙抛了$101$次. 求事件$E:=$\{乙抛出的正面次数比甲多\}的概率.
	\end{exam}

	\begin{jieda}
		\begin{itemize}[<+-|alert@+>]
			\item 如果甲和乙都抛掷这枚均匀的硬币$100$次, 那么当然会有三种不同的可能结果
			\begin{itemize}[<+-|alert@+>]
				\item $A_0$:\pause 甲乙抛出的正面次数一样多\pause
				\item $A_1$:\pause 甲抛出的正面次数比乙多\pause
				\item $A_2$:\pause 乙抛出的正面次数比甲多\pause
				\item $P(A_0)+P(A_1)+P(A_2)=1$\pause 且 $P(A_1)=P(A_2)$\pause
			\end{itemize}
			\item 以乙第一次抛出的硬币结果对样本空间进行分割:
			\begin{itemize}[<+-|alert@+>]
				\item 若$B=$\{乙第一次时抛出的是正面\}发生, 则\pause 乙只要在接下来的$100$次抛掷中, 抛出的正面次数不比甲少即可, 即
				\[P(E|B)=P(A_0)+P(A_1)\]
				\item 若$\overline{B}:=$\{乙第一次时抛出的是反面\}发生, 则\pause 乙只要在接下来的$100$次抛掷中, 抛出的正面次数比甲多即可, 即$P(E|\overline{B})=P(A_1)=P(A_2)$
			\end{itemize}
			\item $P(E)=P(B)P(E|B)+P(\overline{B})P(E|\overline{B})=\frac{1}{2}\left(P(A_0)+P(A_1)+P(A_2)\right)=\frac{1}{2}.$
		\end{itemize}
	\end{jieda}
\end{frame}




\begin{frame}{对战获胜概率问题}
	%	\begin{itemize}
		%		\item 全概率公式是一件有力的工具, 灵活地运用它往往会带来简洁有效的解法
		%	\end{itemize}
	\begin{exam}
		甲、乙进行某项对战比赛, 每回合胜者得$1$分, 败者不得分. 比赛进行到有 1 人比另外 1 人多$2$分终止, 多$2$分者获胜. 现知每回合甲胜的概率为$p\in (0,1)$. 试求$A:=$\{甲最终获胜\}的概率.
	\end{exam}

	\begin{jieda}
		\textcolor{red}{法一(经典解法)}
		\begin{itemize}[<+-|alert@+>]
			\item 显然甲只能在偶数个回合后获胜
			\item 记$A_{2n}$=\{甲在$2n$个回合后获胜\}, \pause 则$A=\cup_{n=1}^\infty A_{2n}$\pause
			\item $A_{2n}, n\geq 1$两两互不相容且\pause $P(A_{2n})=(2p(1-p))^{n-1}p^2$
			\item 甲最终获胜的概率
			\begin{align*}
				P(A)&=\sum_{n=1}^{\infty}P(A_{2n})=p^2\sum_{n=1}^{\infty}(2p(1-p))^{n-1}\\
				&=p^2\sum_{n=0}^{\infty}(2p(1-p))^{n}=\dfrac{p^2}{1-2p(1-p)}.
			\end{align*}
		\end{itemize}
	\end{jieda}
\end{frame}

\begin{frame}{对战获胜概率问题}
	\begin{jieda}
		\textcolor{red}{法二(全概率公式)}
		\begin{itemize}[<+-|alert@+>]
			\item 以前两个回合的战绩对样本空间进行分割
			\item 分别以$B_1,B_2,B_3$表示甲二胜、一胜一败、二败事件
			\item $P(B_1)=p^2,\,P(B_2)=2p(1-p),\,P(B_3)=(1-p)^2$
			\item $P(A|B_1)=1,\,P(A|B_2)=P(A),\,P(A|B_3)=0$
			\item 由全概率公式得$P(A)=\sum\limits_{i=1}^{3}P(B_i)P(A|B_i)=p^2+2p(1-p)P(A)$
			\item $P(A)=\dfrac{p^2}{1-2p(1-p)}$
		\end{itemize}
	\end{jieda}
\end{frame}

\begin{frame}{对战获胜概率问题}
	\begin{jieda}
		\textcolor{red}{法三(随机游动)}
		\begin{itemize}[<+-|alert@+>]
			\item 考察质点在数轴整数点上的随机游动:在整数点$x=n$
			\begin{itemize}[<+-|alert@+>]
				\item 以概率$p$向右移动到整数点$x=n+1$,
				\item 以概率$1-p$向左移动到整数点$x=n-1$
			\end{itemize}
			\item 以$p_n$表示“质点由$x=n$出发, 未达$-2$前先到达$2$的概率”
			\item 显然, $p_{-2}=0,p_2=1$
			\item 甲获胜的概率即为: $p_0$
			\item 由全概率公式易得
			\begin{equation}
				\left\{
				\begin{aligned}
					&p_0=pp_1+(1-p)p_{-1}, \pause \\ \notag
					&p_1=pp_2+(1-p)p_0=p+(1-p)p_0, \pause \\
					&p_{-1}=pp_0+(1-p)p_{-2}=pp_0.\pause
				\end{aligned}
				\right.
			\end{equation}\pause
			\item $p_0=p(p+(1-p)p_0)+(1-p)(pp_0)\pause =p^2+2p(1-p)p_0$\pause
			\item 故 $$p_0=\dfrac{p^2}{1-2p(1-p)}.$$
		\end{itemize}
	\end{jieda}
\end{frame}





\subsection{Bayes 公式的应用}
\begin{frame}{随机抛硬币问题}
	\begin{exam}({\tc 随机抛硬币}) 假设有一枚均匀的硬币和一枚以概率$3/4$正面朝上的不均匀硬币. 随机选取一枚硬币掷$3$次, $3$次都是正面朝上. 试问: %
		\vspace{-0.4cm}
	\begin{enumerate}[<+-|alert@+>]
		\item 给定上述信息后, 选取的硬币是均匀硬币的概率有多大?
		\item 接下去掷第四次时, 仍是正面朝上的概率是多少?
		\end{enumerate}
	\end{exam}
	\pause
%\vspace{-0.2cm}
\begin{jieda}
  令 \begin{align*}
	A &:=\{\mbox{选取的硬币掷 3 次均正面朝上}\}\\
	F & :=\{\mbox{选取的硬币是均匀的}\},\quad
	H :=\{\mbox{第 4 次正面朝上} \}
  \end{align*}
  则\pause
  {\small \begin{align*}
	P(F|A) \pause &=\frac{P(A|F)P(F)}{P(A)}\pause =\frac{P(A|F)P(F)}{P(A|F)P(F)+P(A|F^c)P(F^c)} \\
	\pause &=\frac{(1/2)^3 \cdot 1/2}{(1/2)^3 \cdot 1/2 +(3/4)^3 \cdot  1/2}\pause \approx 0.23\\
    P(H\mid A)\pause & =P(\left.H\mid F,A\right)P(\left.F\mid A\right)+P(\left.H\mid F^{\mathrm{c}},A\right)P(\left.F^{\mathrm{c}}\mid A\right)  \\
	\pause &=\frac{1}{2}\cdot0.23+\frac{3}{4}\cdot(1-0.23) \pause \approx 0.69.
  \end{align*}}
\end{jieda}



\end{frame}







\begin{frame}{罕见病检测诊断问题}
	\begin{exam}考虑以验血结果诊断某种罕见病的患病概率:
		\begin{itemize}[<+-|alert@+>]
			\item 	某罕见病的发病率为$1\%$
			\item 通过验血诊断该病的误诊率为$5\%$, 即非患者中有$5\%$的人验血结果为阳性, 患者中有$5\%$的人验血结果为阴性
			\item 现已知某人验血结果为阳性, 试求他患有此病的概率
		\end{itemize}
	\end{exam}
	\pause

	\begin{jieda}
		\begin{itemize}[<+-|alert@+>]
			\item 记$D:=$\{患有此病\}, $T_{1}:=$\{第一次验血结果为阳性\}.
			\item 要求的概率是: $P(D|T_{1})=\dfrac{P(DT_{1})}{P(T_{1})}=\dfrac{P(D)P(T_{1}|D)}{P(T_{1})}$
			\item 由题意可知%知条件知
			\begin{align*}
				P(D)&=1\%,\,P(\overline{D})=99\%, \, P(T_{1}|D)=95\%,\,P(T_{1}|\overline{D})=5\%\pause \\
				P(T_{1})&=P(D)P(T_{1}|D)+P(\overline{D})P(T_{1}|\overline{D})=1\%\cdot 95\%+99\%\cdot 5\% \pause \\
        &=0.0685\pause
			\end{align*}
		\end{itemize}
	\end{jieda}
\end{frame}

			\begin{frame}{罕见病检测诊断问题}
				\begin{itemize}[<+-|alert@+>]

			\item
			%		$D$和$\overline{D}$构成了对$\Omega$的分划, 故由 T_{1}ayes 公式得
			$P(D|T_{1})=\dfrac{P(D)P(T_{1}|D)}{P(T_{1})}=\dfrac{1\%\cdot 95\%}{0.0685}\approx 0.16.$
		    \item 几率方法:
			$$\frac{P(D|T_1)}{P(D^c|T_1)}=\frac{P(D)}{P(D^c)}\frac{P(T_1|D)}{P(T_1|D^c)}=\frac{1}{99}\cdot\frac{0.95}{0.05}\approx 0.19$$
			\pause
			由几率与概率之间的关系可知
			$$P(D|T_1)=0.19/(1+0.19)\approx 0.16, $$ 与上述结果一致.
			\item 用几率形式的贝叶斯准则计算更迅速的原因是此时不需要计算普通贝叶斯准则的分母.
		\end{itemize}
\end{frame}


\begin{frame}{罕见病检测诊断问题图示}
	\begin{figure}%[ch1-bayes-disease.png]
		\centering
		\includegraphics[width=10.5cm]{figures/ch1-bayes-disease.png}
	  \end{figure}

\end{frame}

\begin{frame}{罕见病检测诊断问题}
	\begin{exam} 接上例, 检测结果为阳性的某人, 决定进行第二次检测. 假设新的检测结果与之前的结果相互独立, 且有相同的敏感性和特异性. 若第二次检测结果也为阳性, 试求此人患有此病的概率.
	\end{exam}
	\pause

	\begin{jieda}
		\begin{itemize}[<+-|alert@+>]
			\item 记 %$D:=$\{患有此病\}, $T_{1}:=$\{第一次验血结果为阳性\}, \\ \quad
			$T_{2}:=$\{第二次验血结果为阳性\}
			\item 要求的概率是: $P(D|T_{1}T_{2})$%$=\dfrac{P(DT_{1})}{P(T_{1})}=\dfrac{P(D)P(T_{1}|D)}{P(T_{1})}$
			\item 一步法: 将两个检测结果一次性都考虑在内以进行概率更新,
			\begin{itemize}[<+-|alert@+>]
			\item 计算几率
            $$\frac{P(D|T_1\cap T_2)}{P(D^c|T_1\cap T_2)}=\frac{P(D)}{P(D^c)}\frac{P(T_1\cap T_2|D)}{P(T_1\cap T_2|D^c)}=\frac{1}{99}\cdot \frac{0.95^2}{0.05^2}=\frac{361}{99}\approx 3.646$$
			\item $P(D|T_{1}T_{2})=\frac{3.646}{1+3.646}\approx 0.78$.
			\end{itemize}
		\end{itemize}
	\end{jieda}
\end{frame}

\begin{frame}{罕见病检测诊断问题}
			\begin{itemize}[<+-|alert@+>]
			\item 两步法:
			\begin{itemize}[<+-|alert@+>]
			\item 在完成第一次检测后, 某人患有此病的后验几率为
             $$\frac{P(D|T_1)}{P(D^c|T_1)}=\frac{1}{99}\cdot\frac{0.95}{0.05}\approx 0.19$$
			 \item 将后验几率作为新的先验几率, 然后基于第二次检测结果更新后验几率%更新为
			 \begin{align*}
				\frac{P(D|T_1\cap T_2)}{P(D^c|T_1\cap T_2)}
				&=\frac{P(D|T_1)}{P(D^c|T_1)}\frac{P(T_2|D,T_1)}{P(T_2|D^c,T_1)}\\
				&=(\frac{1}{99}\cdot\frac{0.95}{0.05})\frac{0.95}{0.05}\\
				&=\frac{361}{99}\approx 3.646
			 \end{align*}

			\item $P(D|T_{1}T_{2})=\dfrac{3.646}{1+3.646}\approx 0.78$
			\end{itemize}

		\end{itemize}
\end{frame}







%\begin{frame}{Bayes 公式的应用}
%	\begin{itemize}
	%		\item 这个结果出人意料的小, 其原因在于人群中该病的患病率很低, 仅为$0.5\%$, 所以尽管通过验血诊断该病的误诊率不算高, 为$5\%$ , 但与患病率相比己是$10$倍之多
	%		\item 当验血结果为阳性时, 确患有此病的概率并不一定就很大
	%		\item 患病的概率除了依赖于验血时的准确率之外, 还与人群中该病的患病率有关, 这一点对于罕见病的诊断尤为重要
	%	\end{itemize}
%\end{frame}
\begin{frame}{电邮问题}
	\vspace{-0.2cm}
	\begin{exam}
		甲、乙二人之间经常用 E-mail 相互联系, 他们约定在收到对方信件的当天即给 E-mail 回复. 由于线路问题, 每$n$份 E-mail 中会有$1$份不能在当天送达收件人. 甲在某日发了$1$份 E-mail 给乙, 但未在当天收到乙的回音. 试求乙在当天收到了甲发给他的 E-mail 的概率.
	\end{exam}
	\pause

	\begin{jieda}
		\begin{itemize}[<+-|alert@+>]
			\item 在这个问题中, 包含了两个不确定的环节:
			\begin{itemize}[<+-|alert@+>]
				\item 甲发给乙的 E-mail 不一定在当天到达乙处
				\item 乙回给甲的 E-mail 不一定在当天到达甲处
			\end{itemize}
			\item $A$=\{乙在当天收到甲的 E-mail\}, $B$=\{甲在当天收到乙回的 E-mail\}
			\item $P(A|\overline{B})=\dfrac{P(A)P(\overline{B}|A)}{P(A)P(\overline{B}|A)+P(\overline{A})P(\overline{B}|\overline{A})}$
			\item %$A$和$\overline{A}$构成了对$\Omega$的分划.
			由题中条件知
			$$P(A)=\frac{n-1}{n},\, P(\overline{A})=\frac{1}{n}, \,P(\overline{B}|A)=\frac{1}{n},\,P(\overline{B}|\overline{A})=1.$$
			\item $P(A|\overline{B})=\dfrac{\frac{n-1}{n}·\frac{1}{n}}{\frac{n-1}{n}·\frac{1}{n}+\frac{1}{n}·1}=\dfrac{n-1}{2n-1}<1/2$
		\end{itemize}
	\end{jieda}
\end{frame}
%\begin{frame}{Poly\'{a}罐子模型续}
%	\begin{exam}
%		罐中放有$a$个白球和$b$个黑球, 每次从罐中随机抽取一个球, 并连同$c$个同色球一起放回罐中, 如此反复进行. 试证明: 在第$n$次取球时取出白球的概率为$\dfrac{a}{a+b}$.
%	\end{exam}
%
%	\begin{jieda}
%		记$A_k$=\{在第$k$次取球时取出白球\}, 于是$A_k^c$=\{在第$k$次取球时取出黑球\}. 以下利用数学归纳法:
%
%		显然有$P(A_1)=\dfrac{a}{a+b}$成立. 假设$n=k-1,\,k\geq 2$时结论成立, 要证$n=k$时结论也成立.
%
%		以$A_1$和$A_1^c$作为对$\Omega$的一个分划, 此时可将$P(A_k|A_1)$看成从原来放有$a+c$个白球和$b$个黑球的罐中按规则取球, 并且在第$k-1$次取球时取出白球的概率, 因此由归纳假设知$P(A_k|A_1)=\dfrac{a+c}{a+b+c}$, 同理亦有$P(A_k|A_1^c)=\dfrac{a}{a+b+c}$.
%	\end{jieda}
%\end{frame}
%
%\begin{frame}{Poly\'{a}罐子模型续}
%	\begin{jieda}
%		于是由全概率公式得
%		\begin{align*}
%			P(A_k)&=P(A_1)P(A_k|A_1)+P(A_1^c)P(A_k|A_1^c)\\
%			&=\frac{a}{a+b}·\frac{a+c}{a+b+c}+\frac{b}{a+b}·\frac{a}{a+b+c}=\frac{a}{a+b}.
%		\end{align*}
%		因此, 结论对一切$n$成立.
%	\end{jieda}
%
%	\begin{itemize}
%		\item 本题解答中对$\Omega$的分划的选取方式值得注意
%		\item 易走的一条歧路是把$A_{k-1}$和$A_{k-1}^c$作为对$\Omega$的分划
%		\begin{itemize}
%			\item 在这种选取之下, 难以利用归纳假设算出条件概率$P(A_k|A_{k-1})$和$P(A_k|A_{k-1}^c)$, 因为此时只知道罐中有$a+b+(k-1)c$个球, 而难以知道白球和黑球的数目
%			\item 相反地, 在$A_1$和$A_1^c$发生的情况下, 罐中白球和黑球的数目十分清楚
%		\end{itemize}
%		\item 这个事实再次表明正确选取分划方式的重要性
%		\item 当然也要正确理解归纳假设
%	\end{itemize}
%\end{frame}

\subsection{条件化及计算概率的递推方法}

\begin{frame}{计算概率的递推方法: 一步分析}

 \begin{exam}{\tc (分支过程)}
池塘里只有一只变形虫叫作 Bobo.
\begin{itemize}[<+-|alert@+>]
\item  每过$1$分钟, Bobo 有三种结果: 死去、分裂成两个或保持原状
\item 三种结果出现的概率相同
\item  此后所有活着的 Bobo 都将继续以这种方式相互独立地进行下去
\item  那么这个变形虫种族最终灭亡的概率是多少?
\end{itemize}
\end{exam}
\vspace{-0.2cm}
     \begin{figure}[分支过程.png]
      \centering
      \includegraphics[width=6cm]{figures/分支过程.png}
    \end{figure}
\end{frame}

\begin{frame}{分支过程(续)}
\begin{jieda}
\begin{itemize}[<+-|alert@+>]
\item 令$D:=\{\mbox{最终种族灭绝}\}$, 本题希望求出$P(D)$.
\item 我们在第一步结果的基础上即以$1$分钟后的结果进行分析:
\begin{itemize}[<+-|alert@+>]
\item 令$B_i:=\{1\mbox{分钟后}{\rm Bobo}\mbox{变成的变形虫个数}  \}(i=0,1,2)$
\item 易知 $P(D|B_0)=1$和$P(D|B_1)=P(D)$, $P(D|B_2)=P(D)^2$.
\end{itemize}
\item 利用全概率公式有
\begin{equation*}
    \begin{aligned}
    P(D)&=P(D|B_0)\cdot\frac{1}{3}+P(D|B_1)\cdot\frac{1}{3}+P(D|B_2)\cdot\frac{1}{3}\\
&=1\cdot\frac{1}{3}+P(D)\cdot\frac{1}{3}+P(D)^2\cdot\frac{1}{3}.
\end{aligned}
\end{equation*}
\item 由上式可解得$P(D)=1$, 即变形虫种族最终会以概率$1$灭绝.
\end{itemize}
\end{jieda}
%\bf{一步分析策略}}在这里是适用的, 因为这个问题在本质上是自相似的: 当 Bobo 保持不变或是分裂成两个时, 都只是原始问题的另一个或另两个复制而已.

\end{frame}







\begin{frame}{对战相遇概率问题}
	\begin{exam}
		包括甲、乙二人在内的$2^n$名乒乓球运动员参加一场淘汰赛.
		\begin{itemize}[<+-|alert@+>]
			\item 第一轮任意两两配对比赛, 然后$2^{n-1}$名胜者再任意两两配对进行第二轮比赛, 如此下去, 直至第$n$轮决出一名冠军为止
			\item 假定每一名运动员在各轮比赛中胜负都是等可能的
			\item 求$B:=$\{甲、乙二人在淘汰赛中相遇\}的概率
		\end{itemize}
	\end{exam}


\end{frame}
\begin{frame}{对战相遇概率问题}
		\begin{jieda}
		记$p_n:=P(B)$, 即甲、乙二人在$2^n$人参赛的比赛中相遇的概率%, 并记他们在第一轮比赛中就相遇的概率为$q_n$.
		\begin{itemize}[<+-|alert@+>]
			\item 以甲、乙二人是否在第一轮比赛相遇对样本空间进行分割
			\item $A:=$\{甲、乙二人在第一轮比赛中相遇\}
			\item 由全概率公式
			\begin{align*}
				p_{n+1}\pause &=P(A)P(B|A)+P(\overline{A})P(B|\overline{A})=\pause P(A)+\pause (1-P(A))\cdot \dfrac{1}{4}p_n\pause
			\end{align*}
			\item $P(A)$: 甲、乙二人在第一轮比赛中相遇的概率
			\begin{itemize}[<+-|alert@+>]
				\item 采用无编号分组模式考虑
				\item $2^{n+1}$个人两两配对的方式一共有$\dfrac{(2^{n+1})!}{2^{2^n}(2^n)!}$种
				\item 甲、乙二人配为一对的配对方式有$\dfrac{(2^{n+1}-2)!}{2^{2^n-1}(2^n-1)!}$种
				\item 将上述两式相除, 即得$P(A)=\dfrac{1}{2^{n+1}-1}$%\\(也可先固定甲, 把其余$2^{k+1}-1$个位置作为变换了的概率空间, 直接得出$q_{k+1}$)
			\end{itemize}
		\item 	$p_{n+1}=\frac{1}{2^{n+1}-1}+\frac{1}{4}(1-\frac{1}{2^{n+1}-1})p_n$
		\item $p_{n+1}=\frac{1}{2^n}$
		\end{itemize}
%		下面对$n$进行讨论.
%
%		如果$n=1$, 显然$p_1=q_1=1$.
%
%		如果$n=2$, 则由包括甲、乙二人在内的$4$个人参加比赛. 分别以$A$和$B$记他们在第一轮和第二轮比赛中相遇的事件, 于是有
%		$$p_2=P(A)+P(\overline{A}B)=P(A)+P(\overline{A})P(B|\overline{A})=q_2+(1-q_2)P(B|\overline{A}).$$
	\end{jieda}
\end{frame}








%\begin{frame}{全概率公式的应用}
%	\begin{jieda}
%		显然, 甲、乙二人在第一轮相遇, 当且仅当他们在第一轮中配为一对, 于是$q_1=\dfrac{1}{3}$; 如果甲、乙二人在第一轮中没有相遇, 那么当且仅当他们在两人都战胜了对手进入第二轮比赛时会相遇, 于是$P(B|\overline{A})=\dfrac{1}{4}$. 如此便知
%		$$p_2=q_2+(1-q_2)P(B|\overline{A})=\frac{1}{3}+\frac{2}{3}·\frac{1}{4}=\frac{1}{2}.$$
%		我们有理由猜测: 对一切$n$, 都应当有$p_n=\dfrac{1}{2^{n-1}}$.
%
%		现利用归纳法证明之. 当$n=1$和$n=2$时, 结论已经成立. 假设$p_k=\dfrac{1}{2^{k-1}}$, 来看$n=k+1$的情形. 仍分别以$A$和$B$记甲、乙二人在第一轮和后续比赛中相遇的事件, 于是有
%		$$p_{k+1}=P(A)+P(\overline{A}B)=P(A)+P(\overline{A})P(B|\overline{A})=q_{k+1}+(1-q_{k+1})P(B|\overline{A}).$$
%	\end{jieda}
%\end{frame}
%
%\begin{frame}{全概率公式的应用}
%	\begin{jieda}
%		由于甲、乙二人在第一轮相遇, 当且仅当他们在第一轮中配为一对. 采用无编号分组模式考虑, 知$2^{k+1}$个人两两配对的方式一共有$\dfrac{(2^{k+1})!}{2^{2^k}(2^k)!}$种, 其中甲、乙二人配为一对的配对方式有$\dfrac{(2^{k+1}-2)!}{2^{2^k-1}(2^k-1)!}$种. 将上述两式相除, 即得$q_{k+1}=\dfrac{1}{2^{k+1}-1}$. \\(也可先固定甲, 把其余$2^{k+1}-1$个位置作为变换了的概率空间, 直接得出$q_{k+1}$)
%
%		如果甲、乙二人在第一轮比赛中没有相遇, 那么欲他们在后续的比赛中相遇, 就必须他们二人在第一轮比赛中双双战胜对手. 而从这时开始便是$2^k$名运动员按照原来的比赛规则进行比赛. 所以只要甲、乙二人都能进入后续的比赛, 那么他们在后续比赛中相遇的概率就是$p_k$, 所以有$P(B|\overline{A})=\dfrac{1}{4}p_k$.
%	\end{jieda}
%\end{frame}
%
%\begin{frame}{全概率公式的应用}
%	\begin{jieda}
%		于是结合归纳假设即知
%		\begin{align*}
%			p_{k+1}&=q_{k+1}+(1-q_{k+1})P(B|\overline{A})=q_{k+1}+(1-q_{k+1})\frac{1}{4}p_k\\
%			&=\frac{1}{4}p_k+\left(1-\frac{1}{4}p_k\right)q_{k+1}=\frac{1}{2^{k-1}}+\left(1-\frac{1}{2^{k-1}}\right)\frac{1}{2^{k-1}-1}=\frac{1}{2^k}.
%		\end{align*}
%		所以结论在$n=k+1$时仍然成立.
%
%		综合上述知, 甲、乙二人在比赛中相遇的概率为$p_n=\dfrac{1}{2^{n-1}}$.
%	\end{jieda}
%	\begin{itemize}
%		\item 以“甲、乙二人是否在第一轮相遇”作为对$\Omega$的分划不仅有利于处理$n=2$的情形, 而且有利于运用归纳假设进行过渡
%	\end{itemize}
%\end{frame}



%\begin{frame}
%	\begin{itemize}
%		\item Simpson 悖论: 光凭直觉是难以作出判断的
%		\item 我们所看到的治愈率都只是些条件概率, 是在己知患者的疾病类型的情况下, 统计出来的治愈率
%		\item 一旦加入了不同疾病人数所占的比例, 就排除掉了这个因素所造成的影响, 得到了不受疾病类型影响的全面的治愈率
%		\item 从这个意义上去评价两种不同的治疗方案, 我们获得了一种全新的视角
%	\end{itemize}
%\end{frame}



%\begin{frame}
%  \frametitle{摸彩模型}
%  \begin{exam}
%    设在$n$张彩票中有一张奖券, 求第二人摸到奖券的概率.
%  \end{exam}
%
%  \jieda 记$A_i:=\{\mbox{第}i\mbox{个人摸到奖券}\}, i=1,2,\cdots,n$, 现在目的是求$P(A_2)$.\pause
%  \begin{eqnarray*}
%    P(A_2)&=&P(A_1)P(A_2|A_1)+P(\overline{A}_1)P(A_2|\overline{A}_1)\\
%    \pause &=&\dfrac{1}{n}\cdot 0 + (1-\dfrac{1}{n})\cdot \dfrac{1}{n-1}\\
%    \pause &=&\dfrac{1}{n}
%  \end{eqnarray*}
%  \pause
%  \begin{eqnarray*}
%    P(A_k)&=&\pause P(\cap_{i=1}^{k-1}\overline{A}_i A_k) \pause =P(\cap_{i=1}^{k-1}\overline{A}_i)P(A_k|\cap_{i=1}^{k-1}\overline{A}_i)\\
%    \pause &=&\cdots = \pause P(\overline{A}_1)P(\overline{A}_2|\overline{A_1})\cdots P(\overline{A}_{k-1}|\cap_{i=1}^{k-2}\overline{A}_i)P(A_k|\cap_{i=1}^{k-1}\overline{A}_i)\\
%     &=&\pause \frac{n-1}{n}\pause\cdot\frac{n-2}{n-1}\pause\cdots \frac{n-k+1}{n-k+2}\pause \cdot\frac{1}{n-k+1}\\
%     &=&\pause \frac{1}{n}
%  \end{eqnarray*}
%
%\end{frame}
%

%\begin{frame}
%  \frametitle{一般摸球模型}
%  \begin{exam}
%    袋中有$r$个红球与$b$个黑球. 每次从袋中任摸出 1 球并连同$s$个同色球一起放回. 以$R_n$表示第$n$次摸出红球, 试证$P(R_n)=\dfrac{r}{r+b}$.
%  \end{exam}
%
%  \pause
%  \zheng 利用归纳法来证明:$n=1$时, $P(R_1)=\dfrac{r}{r+b}$显然成立.
%
%  \pause 假设$n-1$时命题成立. 为求$P(R_n)$, 我们以第 1 次取球的可能结果$R_1$与$\bar{R}_1$作为$\Omega$的分割, 用全概率公式可得:
%  \pause
%  \begin{eqnarray*}
%    P(R_n)&=&\pause P(R_1)P(R_n|R_1)+P(\bar{R}_1)P(R_n|\bar{R}_1)\\
%          &=& \pause\dfrac{r}{r+b}\cdot \dfrac{r+s}{r+s+b}+\dfrac{b}{r+b}\cdot \dfrac{r}{r+b+s}\\
%          &=& \pause\dfrac{r}{r+b}
%  \end{eqnarray*}
%\pause
%\begin{rmk}
%  当$s=0$时相当于放回摸球, 而$s=-1$相当于不放回摸球.
%\end{rmk}
%
%\end{frame}






\begin{frame}
  \frametitle{掷骰子问题}
  % \begin{exam}
  %   % $n$对夫妇在$2n$个一横排椅子上就坐,求事件
  %   % $$A_n=\{\mbox{丈夫全坐在其妻子右方(不一定相邻)}\}$$的概率$p_n$.
  % \end{exam}
  \begin{exam}
   甲乙轮流掷一均匀骰子. 甲先掷,以后每当某人掷出 1 点后则交给对方掷, 否则此人继续掷. 试求事件$A_n=\{\mbox{第}n\mbox{次由甲掷}\}$的概率.
  \end{exam}

  \pause
\jieda 记$p_n=P(A_n)$, 则以$A_{n-}$与$\bar{A}_{n-1}$为分割用全概率公式可得:\pause
\begin{align*}
  p_n=P(A_n)&=\pause P(A_{n-1})P(A_n|A_{n-1})+P(\bar{A}_{n-1})P(A_n|\bar{A}_{n-1})\pause\\
            &=\pause p_{n-1}\dfrac{5}{6}+(1-p_{n-1})\dfrac{1}{6}=\pause \dfrac{2}{3}p_{n-1}+\dfrac{1}{6}\pause
\end{align*}
 经过整理, 可将上式化为以下递推的形式\pause
$$p_n-\frac{1}{2}=\frac{2}{3}\left(p_{n-1}-\frac{1}{2}\right),\,n=2,3,\cdots.$$\pause
由$p_1=1$可得$p_n-\frac{1}{2}=\left(\frac{2}{3}\right)^{n-1}\left(p_1-\frac{1}{2}\right)=\frac{1}{2}\left(\frac{2}{3}\right)^{n-1}.$
\pause 因此, 我们有\pause
\begin{eqnarray*}
  p_n=P(A_n)=\pause\dfrac{1}{2}\bigg[1+\bigg(\dfrac{2}{3}\bigg)^{n-1}\bigg], n=1,2, \cdots,
\end{eqnarray*}

\end{frame}



\begin{frame}{结绳问题续}
	\begin{exam}
$n$根绳$2n$个头两两相接,求$A_n=\{\mbox{恰好结成}n\mbox{个圈}\}$的概率.
	\end{exam}

	\begin{jieda}
		此前曾用条件概率解过本题, 现利用全概率公式给出一个解答.
	\begin{itemize}[<+-|alert@+>]
		\item 将$n$根短绳作编号并记$p_n=P(A_n)$
		\item 记$B$=\{$1$号绳连成$1$个圈\}并用$B$和$\overline{B}$作为对$\Omega$的分划
		\item 由全概率公式可知
		$$p_n=P(A_n)=P(B)P(A_n|B)+P(\overline{B})P(A_n|\overline{B})$$
		\item $P(B)=\dfrac{1}{2n-1}, P(A_n|\overline{B})=0, P(A_n|B)=P(A_{n-1})=p_{n-1}$
		\item 	$p_n=P(A_n)=\frac{1}{2n-1}p_{n-1},\,n=2,3,\cdots.$
		\item 反复利用上式, 并由$p_1=1$可得
		$$p_n=\frac{1}{(2n-1)!!},\,n=1,2,\cdots.$$
	\end{itemize}

	\end{jieda}
\end{frame}



\begin{frame}{秘书问题}
%	\begin{itemize}
%		\item 秘书问题是概率论中的一个著名问题, 它涉及统计试验中的所谓最佳停止时间
%		\item 这类问题很多, 在此仅以秘书问题为例
%	\end{itemize}
	\begin{exam}
		某公司需招收秘书一名, 共有$n$个人报名应聘, 公司面试规则与录取策略如下:
		\begin{itemize}[<+-|alert@+>]
			\item 面试规则: 逐个面试, 并在面试当时对应聘者表态是否录用, 一旦对应聘者表态不录用, 不可改变决定
			\item 录用策略:
			\begin{itemize}[<+-|alert@+>]
				\item 不录用前$k(1\leq k<n)$个面试者
				\item 自第$k+1$个开始, 只要发现某人比他前面的所有面试者都好, 就录用他, 否则就录用最后一个
			\end{itemize}
		\item 试对该公司的策略作概率分析
		\end{itemize}
	\end{exam}
\end{frame}

\begin{frame}{秘书问题}
%	\begin{jieda}
		\begin{itemize}[<+-|alert@+>]
		%	\item 一个关键的问题是如何确定$k$
			\item $A:=$\{最佳人选被录用\}%的事件
			\item$B_j:=$\{最佳人选在面试顺序中排在第$j$位\}
			\item 全概率公式: $P(A)=\sum\limits_{j=1}^{n}P(A|B_j)P(B_j)$%=\dfrac{1}{n}\sum\limits_{j=1}^{n}P(A|B_j)$
			\begin{itemize}[<+-|alert@+>]
			\item $P(B_j)=\frac{(n-1)!}{n!}=\frac{1}{n},\,j=1,\cdots,n$
			\item 当$1\leq j\leq k$时, $P(A|B_j)=0$ \pause (最佳人选位于前$k$个面试者, 不会被录用)\pause
			\item 当$k+1\leq j\leq n$时, $P(A|B_j)=\dfrac{k}{j-1}$ (最佳人选被录用当且仅当前$j-1$个面试者中的最佳者在前$k$个人中)
			\end{itemize}
			\item $P(A)=\dfrac{1}{n}\sum\limits_{j=k+1}^{n}\dfrac{k}{j-1}=\dfrac{k}{n}\sum\limits_{j=k}^{n-1}\dfrac{1}{j}\sim\dfrac{k}{n}\ln\dfrac{n}{k}$
			\item 令$g(x)=\dfrac{x}{n}\ln\dfrac{n}{x},\,x>0$, 则$g'(x)=\dfrac{1}{n}\ln\dfrac{n}{x}-\dfrac{1}{n}=0\Longleftrightarrow x=\dfrac{n}{e}$
			\item 若要$P(A)$达到最大, 只需$k$取最靠近$\frac{n}{e}$的正整数
			\item 最大概率值为$g\left(\frac{n}{e}\right)=\frac{1}{e}\approx 0.36788$与$n$无关
			%\item 即使对很大的$n$, 采用所说的策略, 也能有$0.36788$的概率录用到最佳人选
		\end{itemize}
	%\end{jieda}
\end{frame}









%%% Local Variables:
%%% mode: latex
%%% TeX-master: t
%%% End:
